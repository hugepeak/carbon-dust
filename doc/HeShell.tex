\documentclass[manuscript]{aastex}

\newcommand{\cenn}{{\rm C}_n}
\newcommand{\cnr}{{\rm C}_n^r}
\newcommand{\hep}{{\rm He}^+}

\newcommand{\ctwo}{{\rm C}_2}
\newcommand{\cthree}{{\rm C}_3}
\newcommand{\cfour}{{\rm C}_4}
\newcommand{\cfive}{{\rm C}_5}
\newcommand{\csix}{{\rm C}_6}
\newcommand{\cseven}{{\rm C}_7}
\newcommand{\ceight}{{\rm C}_8}
\newcommand{\ceightr}{{\rm C}_8^r}
\newcommand{\cotoco}{${\rm C} + {\rm O} \to {\rm CO} + \gamma$}
\newcommand{\twoctoctwo}{${\rm C} + {\rm C} \to \ctwo + \gamma$}
\newcommand{\coctoctwo}{${\rm CO} + {\rm C} \to \ctwo + {\rm O}$}
\newcommand{\ctowotococ}{$\ctwo + {\rm O} \to {\rm CO} + {\rm C}$}
\newcommand{\cfivectocsix}{$\cfive + {\rm C} \to \csix$}
\newcommand{\csixotococfive}{${\rm C}_6 + {\rm O} \to {\rm CO} + {\rm C_5}$}
\newcommand{\ncogeq}{n_{CO}^{\gamma eq}}
\newcommand{\nar}{New Astron. Rev.}

\shorttitle{Formation of $\cenn$ Molecules}
\shortauthors{Clayton et al.}

\begin{document}

\title{CARBON CONDENSATION NEAR THE HE SHELL IN SUPERNOVAE}

\author{Donald D. Clayton, Bradley S. Meyer, Lih-Sin The and Tianhong Yu}
\affil{Department of Physics and Astronomy, Clemson University, Clemson, SC 29634-0978}

\begin{abstract}
\end{abstract}

\keywords{atomic processes --- ISM: abundances --- ISM: general --- meteorites, meteors, meteoroids --- supernovae: general --- supernovae: individual (SN 1987A)}

\section{Introduction}

A strategy for this study is to compare the growth of carbon grains from 
three zones near the He-burning shell of Type II supernovae. That shell 
has long drawn attention because incomplete He burning synthesizes more 
carbon than oxygen (Clayton 1968, Ch. 5), yielding throughout the He 
convection zone an abundance ratio C/O $>$ 1 (e.g. Woosley and Weaver 1995) 
that has long been speculated to be of significance for carbon condensation; 
moreover, explosive burning subsequent to the core-collapse shock wave 
yields isotopic features ($e.g. ^{15}$N, $^{18}$O, $^{26}$Al) 
that present important diagnostic clues to the provenance of measured 
isotopic ratios in the large supernova grains extracted from meoteorites 
($e.g.$ Zinner 1998; Clayton and Nittler 2004). 
In fact, this He shell would be the only zone of SNII wherein C condensation 
would be possible if one were to adhere to thermal-equilibrium condensation. 

The rapidly changing supernova environment makes equilibrium condensation 
doubtful and seems to necessitate a kinetic theory. In this study we 
will employ the kinetic theory of carbon nucleation and growth by 
Clayton, Liu and Dalgarno (1999), which has been augmented by 
several subsequent studies (Clayton, Denealt and Meyer 2001; 
Deneault, Heger and Clayton 2003; Deneault, 
Clayton and Meyer 2006; Clayton 2013; 
Yu, Meyer and Clayton 2013). 
That kinetic model of nucleation involves (1) a nearly steady-state 
concentration of linear carbon chains $\cenn$ in the expanding gas, 
(2) transitions of sufficiently long $\cenn$ to ringed isomers, $\cnr$, 
and (3) kinetic capture of free C atoms by ringed $\cnr$. All occur within 
the rapidly expanding and cooling environment. This kinetic model was 
initially proposed to demonstrate that condensation of carbon within 
oxygen-rich gas (C/O $<$ 1) is possible, even inevitable, but it is also a 
reasonable model within C-rich gas. In this work we will apply that model 
within three nearby zones in the supernovae: (1) in the He-exhausted CO core 
just beneath the He-burning shell; (2) in the He-burning shell; (3) above the 
He-burning convective shell. These experience comparable Compton-electron 
fluxes (which we will calculate) but differ significantly in internal 
C/O abundance ratio: in (1) C/O $<$ 1; in (2) C/O $>>$ 1; in (3) C/O is near 
unity but with much smaller concentrations of both C and O. 

Condensation studies have shown that if C/O $<$ 1 oxidation of carbon chains 
reduces their concentration but does not consume them 
(e.g. Clayton, Deneault and Meyer 2001, Fig.2). 
Their abundances $Y_n$ establish a nearly steady state in which reproduction 
by free C balances the oxidation losses. Consequently the number of free 
C atoms per $\cnr$ ring increases with decreasing C/O bulk ratio, so that 
larger but scarcer graphite grains may result. This startling finding, 
which may explain scarcity of graphite SUNOCONs in comparison with 
bulk supernova carbon, was given the name “population control” 
(Clayton, Deneault and Meyer 2001; Deneault, Clayton and Meyer 2006; 
Clayton 2011). 
To examine its relevance to the three zones described above, we will 
calculate grain growth to terminal size within them and compare results. 

A second goal is to understand consequences of the time dependence of the 
fluxes of Compton electrons in these three zones. The buildup locally of 
the intensity of the flux of scattered gamma rays depends on the optical 
depth between the field point and the distribution of radioactive $^{56}$Co. 
The most significant chemical consequences of these energetic electrons lie 
in the creation of $\hep$ and the direct dissociation of CO molecules. These 
chemical effects are time dependent. To study this time dependence we 
calculate the energy deposition by the resulting Compton electrons as a 
function of time in three differing models of the postsupernova that 
differ in their degree of outward transport of radioactive $^{56}$Ni, so 
that the lifetimes of He and of CO can be treated in a time-dependent 
manner within the chemical network. The $\hep$ ion dissociates molecules 
owing to the energetic charge exchange when $\hep$ is neutralized during 
collisions with them. The dissociation of CO molecules by both $\hep$ and 
Compton electrons provides an ongoing source of free carbon that maintains 
the abundances of Cn against oxidation and also feeds the growth of 
larger carbon grains.

\section{The Compton Electron Flux and the Lifetimes of He and CO}

To calculate the lifetimes of He and CO against energetic Compton electrons 
we employ the gamma-ray transport code constructed by 
The, Burrows and Bussard (1990) to obtain the energy deposition 
rate in Compton electrons $\epsilon(t)$ as a function of position and of time 
in the supernova. We then calculate the lifetimes of He and of CO using the 
concept of $energy$ $per$ $ion$ $pair$. The lifetimes are inserted into 
our chemical 
reaction network (Yu, Meyer and Clayton 2013), providing time-dependent source 
terms for the abundance of $\hep$ and for dissociation of CO; namely,
\begin{equation}
\frac {{\rm d}Y(\hep)} {{\rm d}t} =
  \frac {Y(He)}{\tau_e(He)} - {\rm destruction\ rate\ of\ }Y(\hep)
\end{equation}
\begin{equation}
\frac {{\rm d}Y(CO)} {{\rm d}t} =
  -\frac {Y(CO)}{\tau_e(CO)} - 
  \frac {Y(CO)}{\tau_\hep(CO)} + 
  {\rm production\ rate\ of\ }Y(CO)
\end{equation}
where $\tau_e(CO)$ is the lifetime of CO against Compton electrons and 
$\tau_\hep(CO)$ is the lifetime of CO against $\hep$. These lifetimes are 
calculated by us from the radioactive energy deposition rates in three 
models of SNII that differ only in their degree of central confinement of 
radioactive $^{56}$Ni. Our standard SNII is represented by model w10hmm of 
SN1987A (Pinto and Woosley 1988). That model mixed radioactive 
56Ni outward between $M_r$=2 and $M_r$=15 with a declining radial 
concentration in such a way as to reproduce the observed early emergence of 
hard X rays from the surface of SN87A. 
  
We take the spherically symmetric decline with increasing radius of 
the radial  density of $^{56}$Co radioactivity mixed outward to represent 
Pinto and Woosley’s (1988)  spherically symmetric approximation 
of the gamma-ray transport effects of several non spherical fingers of 
$^{56}$Co that were transposed with overlying gas in the radial outward flow. 
The energetic 10-30keV X rays were measured by Japan satellite $Ginga$ 
(Dotani et al. (1987)) in early July 1987, about five months following 
SN87A. Their measurements showed the flux to grow throughout that summer, 
leveling off in September. By mid August the Russian $Mir-Kvant$ observatory 
had detected harder 20-300keV X rays with phoswitch detectors 
(Sunyaev et al. (1987)). These detections agree except below 10keV, 
although emphasizing different portions of the spectrum. Taken together they 
suggest that the hard X ray flux from Compton scattering grew steadily in 
SN87A for the first six months (say from $5\times 10^6$s to $1.5\times 10^7$s). 
Using the w10hmm mixing model enables us to compute $\epsilon(t)$ at the He 
shell as a function of time for that supernova model.

We compare the energy deposition rate in model w10hmm with those from 
otherwise identical models in which $^{56}$Ni concentration has the same 
radial profile out only to $M_r=3.5$, slightly beneath the He burning shell 
(w10hmm3.5), and another in which 56Ni remains confined to central 
$M_r=2.0$ (w10hmmc). The extent of radial mixing governs the rise time 
for energy deposition in the He burning shell. These alterations of 
the mixing profile illustrate models of SNII that differ plausibly from 
SN87A. In each of those three models of SNII we calculate the energy 
deposition rates $\epsilon(t)$ in the three zones of SNII described above. 
Those rates of energy deposition as energies of Compton electrons provide 
$\epsilon(t)$ at the He shell as a function of time. Thus 
$\tau_e(He)$ will also be a 
function of time. That means that $\hep$/He will also be a function of time. 
Because $\hep$ has some important chemical reactions (such as dissociating CO), 
the chemical network takes on a new time dependence that did not exist in 
Yu et al. (2013). We fit, approximately, $\epsilon(t)$ for 10hmm to a simple 
function of time at the He shell. Then $\tau_e(He)$ will express as the same 
function of time, as will also $Y(\hep$). These values can be read into the 
network from the analytic fit. Then we then compare $Y_8^r$ in three 
neighboring 
zones: $M_r=3.8$ just under He-burning shell, $M_r=4.2$ in He-burning shell, 
and $M_r=6.1$ above He-burning shell. These differ interestingly in 
elemental and isotopic compositions. Table (ref1) lists the initial 
compositions of the important chemical elements in these three zones. 
We include N and Ne despite their absence from the current version of our 
chemical network:

That composition lists only five elements, He, C, O, N and Ne. The first 
three are the only elements in our current chemical network 
(Yu et al. 2013), because those three elements govern the formation of 
linear $\cenn$. When performing our chemical network calculations for the 
number of rings $Y_8^r$ formed in those zones, we use the unmixed composition 
of each zone, although Pinto and Woosley (1988, see their Fig. ref1) 
actually mixed the supernova with a prescribed radial concentration 
gradient using spherical symmetry befitting the intended gamma-ray 
propagation code. But we calculate molecular reactions in the unmixed zones 
on the expectation that most of the mass of the zones remains unmixed 
chemically. Our paper will focus on the molecular differences among these 
three zones in the attempt to understand the effects of differing chemical 
compositions and of time-dependent $\tau_e$. 

Characteristic values from our calculations with model 10hmm are given in 
Table ref2:

Our calculations displayed in Fig. ref1 show that $\epsilon(t)$ varies with 
time in proportion to the rate of $^{56}$Co decay, namely exp(-t/111d). 
That exponential line is plotted also in Fig. ref1 through the value 
1.870E+07 ergs/g/s at t=100d:  therefore, at $M_r=5.90$, 
$\epsilon(t)$=1.870E+7exp(-(t-100d)/111d) ergs/g/s. The fit is excellent, 
confirming that no other time dependences are significant in this time range. 
Similar fits are obtained for the two other zones. The absence of 
other time dependences reflects that fact that w10hmm mixed $^{56}$Co 
substantially to beyond the He shell, so that the gamma flux is 
immediately present. The rate of increase of $\epsilon(t)$ at (t=100d) 
between CO 
core and He shell is approximately d$\epsilon(t)$/d$M_r$ = -1.0E+07 ergs/g/s 
per solar mass. These values can be used to calculate the lifetimes 
$\tau_e(He)$ 
and $\tau_e(CO)$ as functions of time and radial mass zone. The rising values 
of $\epsilon(t)$ for t$<$50d are due to still live $^{56}$Ni mixed out to 
these zones; but we can ignore that time dependence because for t$<$50d the 
model temperature is too hot for stable molecules. 

\acknowledgments

T. Y. gratefully acknowledges the support of
a NASA Earth and Space Science Fellowship.
This work was also supported by NASA grant NNX10AH78G.

\end{document}
