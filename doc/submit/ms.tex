\documentclass[manuscript]{aastex}

\newcommand{\ctwo}{{\rm C}_2}
\newcommand{\cthree}{{\rm C}_3}
\newcommand{\cfour}{{\rm C}_4}
\newcommand{\cfive}{{\rm C}_5}
\newcommand{\csix}{{\rm C}_6}
\newcommand{\cseven}{{\rm C}_7}
\newcommand{\ceight}{{\rm C}_8}
\newcommand{\ceightr}{{\rm C}_8^r}
\newcommand{\cenn}{{\rm C}_n}
\newcommand{\cotoco}{${\rm C} + {\rm O} \to {\rm CO} + \gamma$}
\newcommand{\twoctoctwo}{${\rm C} + {\rm C} \to \ctwo + \gamma$}
\newcommand{\coctoctwo}{${\rm CO} + {\rm C} \to \ctwo + {\rm O}$}
\newcommand{\ctowotococ}{$\ctwo + {\rm O} \to {\rm CO} + {\rm C}$}
\newcommand{\cfivectocsix}{$\cfive + {\rm C} \to \csix$}
\newcommand{\csixotococfive}{${\rm C}_6 + {\rm O} \to {\rm CO} + {\rm C_5}$}
\newcommand{\ncogeq}{n_{CO}^{\gamma eq}}
\newcommand{\nar}{New Astron. Rev.}

\shorttitle{Formation of $\cenn$ Molecules}
\shortauthors{Yu et al.}

\begin{document}

\title{Formation of $\cenn$ Molecules in Oxygen-rich Interiors of Type II
Supernovae}

\author{Tianhong Yu, Bradley S. Meyer, and Donald D. Clayton}
\affil{Department of Physics and Astronomy, Clemson University, Clemson, SC 29634-0978}

\begin{abstract}
Two reaction-rate-based kinetic models for condensation of carbon dust via the
growth of precursor linear carbon chains are currently under debate. The first
involved forming $\ctwo$\ molecules via radiative association of free C
atoms; the second forms $\ctwo$\ molecules by the endoergic reaction
\coctoctwo. Both are
followed by C captures until the linear chain eventually makes an isomeric
transition to ringed carbon on which rapid growth of graphite may occur.
These two approaches give vastly different results. Because of the high
importance of condensable carbon for current problems in astronomy,
we study these competing claims with an intentionally limited reaction-rate
network that shows clearly that initiation by
\twoctoctwo\ is the dominant
pathway to carbon rings. We propose an explanation for why the second pathway
is not nearly as effective as its proponents calculated it to be.
\end{abstract}

\keywords{atomic processes --- ISM: abundances --- ISM: general --- meteorites, meteors, meteoroids --- supernovae: general --- supernovae: individual (SN 1987A)}

\section{Introduction}

Whether carbon dust can or can not condense in the interiors of expanding 
Type II supernovae is an important question for astronomy. Many have 
assumed, guided by chemical equilibrium, that carbon condensation can 
not occur unless carbon is more abundant than oxygen. That assumption 
limits C condensation to the He-burning shell of SNII. This rule-of-thumb 
was challenged when \citet{1999Sci...283.1290C}
introduced a non-equilibrium kinetic theory by which carbon can 
condense thermally in gas having more abundant oxygen, opening the 
possibility of C condensation throughout the O-rich core. That 
initial theory has been amplified by several studies 
\citep{2001ApJ...562..480C,2003ApJ...594..312D,2006ApJ...638..234D,
2011NewAR..55..155C, 2013ApJ...762....5C}.
Within their chemical model \citet{2009ApJ...703..642C,2010ApJ...713....1C}
raised questions about the most effective way of producing $\ctwo$\ molecules, 
which is the first step to linear carbon chains. They discovered that 
the neutral-neutral reaction \coctoctwo\ has a reasonably large cross section 
$\langle \sigma v \rangle = k_{CO,C} = 8.6 \times 10^{-14}$ cm$^3$ s$^{-1}$
at temperatures near 5000K despite being endothermic by 4.8eV. 
Although the CO target abundance for production of $\ctwo$\ in this 
manner must be calculated, they stated that in their model 
at $T = 5000$ K it created $\ctwo$\ much more rapidly than the slow radiative 
reaction \twoctoctwo. Sensing high importance of this reaction for astronomy, 
they claimed in their list of conclusions; 2. A new pathway to the 
formation of carbon chains is active in the O-rich mass zone of the 
unmixed ejecta and is identified as the CO conversion to $\ctwo$\ via 
collisions with C. \citet{2009ApJ...703..642C} were unable to evaluate 
the quantitative consequences because their reaction network 
terminated at $\cthree$\ and did not include the formation of linear 
carbon chains $\cenn$\ and their isomerization to carbon rings. 
\citet{2010ApJ...713....1C} subsequently did include 
linear chains to $n=10$, which they took to isomerize instantly 
to ringed structure, following the published 
model \citet{1999Sci...283.1290C, 2001ApJ...562..480C}.

Testing their calculated abundances for $\cenn$\ is but a secondary 
goal of the present study. Its overriding goal is to point out 
abundance differences of several orders of magnitude for the CO 
and $\ctwo$\ molecules between the computations of these two groups 
and to resolve those differences if possible. This test has 
important consequences for condensation of carbon dust, changing 
prior expectations if the production of $\ctwo$\ owing to the reaction 
\twoctoctwo\ severely underestimates the subsequent abundances of 
linear carbon chains. \citet{2010ApJ...713....1C} harbored that 
expectation owing to their belief that much more $\ctwo$\ is 
made earlier near $T = 5000$ K and survives. Dust created by 
supernova expansions is currently under intense study owing to three 
types of astronomical observations: (1) of dust observed in single 
supernova remnants; (2) of dust observed in early low-metallicity 
galaxies; (3) of supernova-condensed carbon dust (SUNOCONs) 
extracted from meteorites. Each of these topics depends sensitively on 
how much carbon, both numbers and sizes, is able to condense in 
cooling SNII interiors. Therefore we study this competition carefully.

Crucial to this task is the lifetime $\tau_{CO}$ of CO molecules. 
Thermal dissociation of CO is dominated by thermal photons because 
radiative association of C and O dominates other reactions for 
the formation of CO at the densities within SNII. Each reaction 
and its inverse is subject to the quantum principle of detailed 
balance. The thermal photodissociation lifetime $\tau_\gamma$ is 
calculated from detailed balance with the radiative association reaction, as in 
Section 2.1 of \citet{2001ApJ...562..480C}. Because of the large 11.1eV 
binding energy of the CO molecule, $\tau_\gamma$ is very temperature sensitive. 
For readers’ numerical ease we will tabulate $\tau_\gamma$ at selected key 
temperatures in Table 1 to follow. The flux and spectrum of newly 
injected Compton electrons \citep{1991ApJ...375..221C} yields the 
lifetime $\tau_e$ of CO molecules against inelastic scattering dissociation by 
Compton electrons caused by the $^{56}$Co radioactivity. In a gas of pure 
CO the mean energy per ion pair is defined as the energy of primary 
electrons divided by the number of pairs produced. 
\citet{1994ApJ...435..909L} calculated the mean energy per ion pair 
in pure CO gas and obtained the result $\Delta E = 32.3$ eV deposited 
per dissociated CO pair, agreeing with the measurement of 32.2 eV 
by \citet{1968...Klots}. So the efficiency of energetic electrons for 
dissociating CO seems well established. A lifetime $\tau_{CO}$ near one 
week is typical in SN 1987A, but would be longer in SNII 
synthesizing less $^{56}$Co and at times greater than about 8 months.

\section{The Physical Model and Network}

Because our goal is to study the chemistry, we can take a very simple physical
model of the expansion; namely, temperature
$T = 3800 K / (t / 100 {\rm d}) =
3.30 \times 10^{10} / t({\rm s})\ {\rm K}$,
where t is the time elapsed since core collapse.
We chooose $n_O = 10^{10}$ cm$^{-3}$
and $n_C = 10^9$ cm$^{-3}$
at the starting time $t = t(6000 {\rm K})$ 
at $T=6000$ K for our chemical network . From our model choice for $T(t)$ we get
$t(6000K) = 5.47 \times 10^6$s. At
subsequent times $n_O(t) = 10^{10}{\rm cm}^{-3}(t/t(6000 {\rm K}))^{-3}$
owing to homologous expansion.
Let $N$ be the number of any specific molecular species
in a comoving, expanding, initially 1 cm$^3$ volume at 6000 K.
The only change of $N$ during
expansion occurs through chemical reactions. $N$ may be expressed as atom
fraction $Y$ of the initial total number
$N_O + N_C = 1.10 \times 10^{10}$ atoms. We
intentionally choose an O-rich interior having $N_O/N_C = 10$,
so that no carbon at
all would be able to condense if that interior were governed by chemical
equilibrium. 

We use an intentionally limited set of chemical species because our goal is
to study the controversy over the correct 
carbon chemistry pathway to the ringed isomers.
We limit the present study to C, O, CO, $\ctwo$, $\cthree$, ... $\ceight$\ and
$\ceightr$, the ringed isomer of $\ceight$, to which we give a lifetime
$\tau_8^r = 10$ s against thermal isomeric transition
from linear $\ceight$ to ringed $\ceightr$.
In a sense the $n = 8$ ring is a dummy standing for all rings; but it is also
reasonable in being the smallest ring that is widely expected 
(e.g. \citealt{doi:10.1021/j100374a025}).
Those ringed molecules are the seeds for carbon
growth because their oxidation rates are much smaller than oxidation rates of
linear $\cenn$
whereas their C capture rates are fast. 
For purposes of this study we take ringed $\ceightr$ to be
indestructible, simply integrating its rate of production.
%We cannot allege that $\ceight$\ is the key isomerizing chain, nor that
%$\tau_8^r = 10$ s is its lifetime, but those assumptions are adequate for
%comparing models for formation for $\ctwo$.
Our strategy is
to compute the number $\ceightr$ remaining after expansion from two differing
sources of $\ctwo$\ initiating the linear $\cenn$
chains. Those two source reactions are
\begin{itemize}
\item \twoctoctwo\ \citep{1999Sci...283.1290C}.
\item \coctoctwo\ \citep{2009ApJ...703..642C,2010ApJ...713....1C}.
\end{itemize}
We take our reaction rates from the rate tables in
\citet{2009ApJ...703..642C,2010ApJ...713....1C}.
The rate for the second reaction is indeed
much greater than that of the first reaction above $T=3000$ K;
but it becomes increasingly the smaller of the two below $T=2500$ K.
We include the thermal inverse reaction of every reaction,
which we calculate as in Eq.(3) of \citet{2001ApJ...562..480C}.
See Table 1 below for sample thermal dissociation rates for CO molecules.
We also include in our network dissociation of CO by Compton electrons.
CO is the only molecule for which electron dissociation can be the dominant
destruction mechanism. For that rate we use
$\tau_e (s) = 10^5 \exp( (t-10^6{\rm s}) / 111 {\rm d} )$,
not fit to any specific model but with plausibility for SN1987A.
It lengthens to $\tau_e = 10^6$ s near eight months.
Because of their helpful cataloging of reaction rate tables,
and in order to compare results without rate differences, we take the
rates as given in the tables of
\citet{2009ApJ...703..642C,2010ApJ...713....1C}.
As one example, we take the rate for \twoctoctwo\
as the rate given by RA4 in Table 5 in \citet{2009ApJ...703..642C}, p. 649.

Our computational reaction network is {\em libnucnet}
(\citealp{2007M&PSA..42.5215M}--see also
\url{http://sourceforge.net/projects/libnucnet/}) modified to follow
the chemical rather than nuclear reactions.
This network code has been thoroughly tested
on a wide variety of reaction networks and problems.
We scrupulously tested our network answers by hand calculations capable of
exposing coding errors.

\section{The Function $n_{CO}(t)$}

The interior core of C and O resulting from completed He burning in massive
stars is a mix of C and O atoms having bulk C/O $<$ 1. Post explosive cooling of
such matter will attempt to associate C and O into CO molecules, with rate
coefficient $k_{CO} = 3.3 \times 10^{-17}$ cm$^3$ s$^{-1}$
\citep{1990ApJ...358..262L}.
The reaction \cotoco\ is one of the crucial reactions of chemical
astrophysics. Its rate $k_{CO}$ is intrinsically slow because quantum mechanics
requires not only rearrangement of electronic shells but also simultaneous
creation of a photon during the collision; nonetheless, the huge product
$n_C n_O$ in the He-exhausted core ensures steady growth for the CO abundance
until it is reversed by radioactive dissociation. 

Evidently the abundance of CO within that zone during expansion attempts to
balance these creation and destruction effects, doing so exactly at the
time $t = t_{max}$ of maximum $n_{CO}$. \citet{2013ApJ...762....5C}
has discussed the shape of the function
$n_{CO}$(t). His Eq.(1), which is valid at constant density,
approximates the growth of $n_{CO}$ by its leading terms:
\begin{equation}
\frac{dn_{CO}}{dt} = n_C n_O k_{CO} - \frac{n_{CO}}{\tau_{CO}} = 0
\label{eq:dncodt}
\end{equation}
at $t = t_{max}$.
The maximum abundance reached by CO is in cgs units
\begin{equation}
n_{CO}(t_{max}) = n_C n_O k_{CO} \tau_{CO} 
\label{eq:comax}
\end{equation}

This amount is equal to that formed during its last mean lifetime $\tau_{CO}$
against dissociation. Eq. (\ref{eq:comax}) is also highly accurate in
circumstances where the time derivative in Eq. (\ref{eq:dncodt}) is but a
small difference between much larger creation and destruction terms. Such
balance is often set up as an abundance approaches its true maximum. 
If $\tau_{CO}$ is taken to be $\tau_\gamma$ because
the dissociation of CO is dominated for $T > 3500$ K by thermal photons,
one obtains the expression for the abundance $n_{CO}^{\gamma eq}$ in thermal
equilibrium: $n_{CO}^{\gamma eq} = n_C n_O k_{CO} \tau_\gamma$,
and is also entered in Table 1. 

Expressed instead in terms of number fractions $Y_{CO} = n_{CO} / n$
and $Y_C = n_C / n$ where $n$ is the number density of all atoms,
Eq. (\ref{eq:dncodt}) transforms to 
\begin{equation}
\frac{dY_{CO}}{dt} = Y_C Y_O n k_{CO} - \frac{Y_{CO}}{\tau_{CO}}	
\label{eq:dycodt}
\end{equation}
which is the form integrated by our network of coupled reactions.
The number density $n$ is needed for the rate
$n({\rm cm}^{-3}) k_{CO}({\rm cm}^3{\rm s}^{-1})$.
In our numerical example we take $n_O = 10^{10}$ cm$^{-3}$ and
$n_C = 10^9$ cm$^{-3}$ at $t(6000 {\rm K})$ as starting conditions for the
chemical network. Subsequent values are $n = 1.1 \times 10^{10}$ cm$^{-3}$
$[t(6000 {\rm K})/t]^3$, where the expansion factor reduces the initial number
density appropriately. Expansion factors are also in Table 1. Then
Eq. (\ref{eq:comax}) reads:
\begin{equation}
Y_{CO}(t_{max}) = Y_C Y_O n k_{CO} \tau_{CO}
\label{eq:ycomax}
\end{equation}
which is valid during expansions.

The lifetime of CO against dissociation is a composite of two physical
reactions. Letting $\tau_\gamma$ be the photodissociation lifetime owing
to thermal photons and $\tau_e$ be the dissociation lifetime owing to fast
Compton electrons, we have
\begin{equation}
\frac{1}{\tau_{CO}} = \frac{1}{\tau_\gamma} + \frac{1}{\tau_e}
\label{eq:tau_co}
\end{equation}
Which partial lifetime dominates the dissociation depends on the temperature.
The radioactive lifetime $\tau_e$ is taken to be
$\tau_e ({\rm s}) = 10^5 \exp( ( t -10^6{\rm s}) / 111 {\rm d} )$,
but the thermal photodissociation lifetime $\tau_\gamma$
depends strongly on temperature
owing to the large binding energy of the CO molecule.
Table \ref{tab:quantities} displays a
short list of $\tau_\gamma$ at key temperatures as well as several related
quantities.

One sees from Table \ref{tab:quantities}
that $\tau_\gamma$ dominates Eq. (\ref{eq:tau_co}) for $T > 3500$ K,
whereas Compton
electron dissociation $\tau_e$
dominates below 3500 K. We start computation of our
chemical network at $T = 6000$ K,
so $n_{CO}(t)$ will initially be small and will
grow as temperature declines owing to the lengthening of $\tau_\gamma$ with
falling temperature. After $t=t_{max}$ the abundance of CO declines owing to
the destruction rate $1/\tau_e$ exceeding the creation rate in
Eq. (\ref{eq:dncodt})
while the gas cools to temperatures at which carbon chains can grow and
isomerize to rings \citep{1999Sci...283.1290C}. Such rings are taken to be
the nucleations upon which graphite grows.

Figure \ref{fig:nco}
compares our network calculation of the abundance of CO molecules
with the expectation of Eq. (\ref{eq:comax}) with $\tau_{CO} = \tau_\gamma$
dominating the destruction of CO.
Solid points display the equilibrium product $\ncogeq$ calculated from the
factors shown in Table \ref{tab:quantities}
and from the number densities $n_C$ and $n_O$ after
their initial values at $T=6000$ K have been reduced by the expansion factor
$(t/t(6000 {\rm K}))^{-3}$.
Tight agreement for $T > 4000$ K is immediately evident,
demonstrating that for $T > 4000$ K the abundance of CO is almost exactly in
thermal equilibrium. Eq. (\ref{eq:comax}) validly describes the black dots in
Fig. \ref{fig:nco} because creation and destruction terms are very nearly
balanced while $T > 4000$ K.
Below 3500 K the abundance of CO becomes much smaller than
the expectation $\ncogeq$ of thermal equilibrium because the dissociation of CO
comes to be dominated by Compton electrons. The dashed vertical line at
$T=3500$ K marks the approximate boundary between these two mechanisms for CO
dissociation. Notice carefully in Fig. \ref{fig:nco}
that $n_{CO}$ grows slowly as $T$ falls, not reaching
its final maximum until expansion has cooled to $T=3500$ K.
Although it is obvious that equilbrium CO increases as $T$ falls,
assuming that to occur begs the question of achieving equilibrium.
Our kinetic results demonstrate that CO does quickly achieve its
equilibrium abundance above $T = 4000$ K, and Eq. (\ref{eq:dycodt})
shows that equilibrium to increase as $\tau_\gamma$ lengthens.
At its final maximum $n_{CO} = 1.5 \times 10^6$ cm$^{-3}$,
corresponding to $Y(CO) = 0.7 \times 10^{-3}$. The slow growth of $n_{CO}$
shown in Fig. \ref{fig:nco} differs markedly from the results of
\citet{2009ApJ...703..642C}.
Their results (e.g. Fig. 11) show $n_{CO}$ climbing quickly near 6000 K
to a large
maximum number fraction near 0.1. That maximum would almost exhaust free
carbon. Our results in Fig. \ref{fig:nco}
so differ from theirs that the difference
must be resolved. \citet{2013ApJ...762....5C} has analyzed the expectation
of growth of $n_{CO}$ to a single maximum before declining, and our results are
in line with that expectation. 

These features are further detailed in Fig. \ref{fig:flows},
which shows the reaction
currents (reactions per second per atom) into and out of CO. There exists a
near steady state in that the production of CO is almost balanced by the
two flows destroying CO. The destruction flow $Y_{CO}/\tau_\gamma$
almost balances the
association flow above 4000K except for a tiny excess leading to the
slow growth of $n_{CO}$ evident in Fig. \ref{fig:nco}. In that temperature 
range the
production by \cotoco\ is in equilibrium with the thermal radiation field,
as Fig. \ref{fig:nco} implied. The decline of the flow creating CO occurs owing
to expansion. The destruction flow $n_{CO}/\tau_e$ almost balances the
creation flow below 3500 K. The transition between destruction modes occurs
near 3500 K. It is no coincidence that the maximum of $n_{CO}$ occurs when
Compton electrons begin to dominate CO dissociation, because $n_{CO}$ would
continue growing as long as thermal photons dominate CO dissociation. 
Fig. \ref{fig:flows} also shows the destructive flow by inelastic electrons
to slightly exceed the production flow for $T < 3500$ K. That modest difference
causes CO to decline following it maximum. 

\section{Abundances of $\cenn$}

Figure \ref{fig:ncoc2} displays the number density $n_{\ctwo}$ along with
that of $n_{CO}$.
Our examination of flows into $\ctwo$\ shows that the reaction
\coctoctwo\ competes with the reaction \twoctoctwo\ only in the
range 3400 K $< T < 3900$ K, but falls steeply for greater or lesser
temperature.  At higher $T$ the abundance of CO is too small to create
$\ctwo$\ in this way, and at smaller $T$ the cross section for \coctoctwo\
declines too precipitously.  \citet{2009ApJ...703..642C} missed that
because their calculated abundance of CO was much too large.
At its maximum $n_{\ctwo} =0.1$ cm$^{-3}$ is only about
$10^{-7}$ of $n_{CO}$. The reason $\ctwo$\ is so rare is that its dissociation
rate by thermal photons is much faster than that for CO because the binding
energy of $\ctwo$\ is
so much less than that for CO.
The smaller value of $\tau_\gamma$ causes the steady state with thermal
radiation for $\ctwo$, as in Eq. (\ref{eq:dycodt}), to be much smaller than 
that for CO, so $n_{\ctwo}$ grows much more slowly than $n_{CO}$.  The Compton
electron lifetime $\tau_e$ never plays a role for $\ctwo$, but the
lifetime against oxidation becomes faster that $\tau_\gamma$ when $T$
falls below 3500 K.  That transition establishes the maximum in
Fig. \ref{fig:ncoc2}.
This $n_{\ctwo}$
maximum differs greatly
from results in \citet{2009ApJ...703..642C}, where their Fig. 11 shows the
maximum atom fraction of $n_{\ctwo}$ to be near $10^{-5}$,
equal to about $10^5$ cm$^{-3}$, almost 1\% of their $n_{CO}$ at that time.
And their maximum for $\ctwo$\ occurs near 7000 K, far too hot for $\ctwo$\ to
be abundant in the face of rapid photodissociation. The huge differences between
these two computations require an explanation. We find it likely that
\cite{2009ApJ...703..642C} inadvertently omitted photodissociation by thermal
photons from their destruction rates.  


Figure \ref{fig:yi}
shows the abundances $Y_i$ of each species in our small network as a
function of time $t-t(6000 {\rm K})$ after $T=6000$ K.
From Table \ref{tab:quantities},
start time is $t(6000K) =5.47 \times 10^6$ s.
To display each $Y_i$ on a figure with
reasonable ordinate resolution, we have scaled each $Y_n$\ by the factor
stated in the box. Many features are noteworthy: (1) $Y_C$ and $Y_O$ 
are constant
because their small depletion is negligible on the scale shown;
(2) maxima of $Y_{CO}$ and $Y_2$\ occur at almost exactly the same time
$t-t(6000K)= 4 \times 10^6$ s, as was also seen in Fig. \ref{fig:ncoc2};
(3) $\cthree$\ has very small abundance, about $10^{-7}$ of $Y_2$, 
although much later $Y_3$\ slowly grows modestly
relative to much more abundant $\ctwo$; (4) the rise shapes of $Y_4$
through $Y_8$\ are very similar because they are linked by a near steady
state; (5) the ringed carbon $Y(\ceightr$) has similar abundance shape vs. time,
but notice that it is actually much more abundant than lineaer $C_{7-8}$ and
because it accumulates from isomeric transitions of $\ceight$\
and unlike $\cenn$\ does not
suffer from fast oxidation \citep{1999Sci...283.1290C}. Each of these
features is understandable in terms of the flows into and out of each species. 

One sees from Fig. \ref{fig:yi} that the \coctoctwo\ reaction plays no
role in $\ceightr$\ production from the fact that $\ceightr$\ rises only for
$t - t(6000 {\rm K}) > 1.5 \times 10^7$ s, corresponding to $T < 2000$ K,
despite $\ctwo$\ and $\cthree$\ rising at much earlier times.
Whatsoever $\ctwo$\ is made earlier plays no role in $\ceightr$\ production
because the ejection of C atoms from $\cenn$\ at higher $T$ prevents the
flow to $\ceight$.  Only at $T < 2000$ K do those photodissociation reactions
become so slow that $\ceight$\ can grow, which it does from the $\ctwo$\
recently
formed by the \twoctoctwo\ reaction near 2000 K.  Fig. \ref{fig:yi} also shows
that $Y(\ceightr)$
grows quickly to $10^{-16}$.  Since available C is $Y_C = 0.1$,
the ratio gives $10^{-15}$ C/ring.  This is enough to grow very large
graphite.

Figure \ref{fig:cn} shows the shape of $\cenn$\ vs. $n$ for $\cthree$\ to
$\ceight$\ at two different temperatures near 2000 K. As temperature declines,
the curve flattens because photoejection from $\cenn$\ by thermal photons
(or vibrations) weakens. These abundance ratios are almost in a steady state,
but that steady state changes slightly as $T$ falls. These patterns show
almost equal values for $Y(\cthree)$ because that curve is relatively flat
with time near $t-t(6000K) = 10^7$ s (Fig. \ref{fig:yi}).
Table \ref{tab:flows} lists our computed reaction flows both in and out of
$\csix$\ at $T = 2031$ K and $T = 1802$ K.
The reactions are specified there in the
compact notation target(in,out)residual. 
We tabulate the flow magnitudes for $\csix$ to aid readers checking our
numerical results. Fig. \ref{fig:cn} and
Table \ref{tab:flows} are made to be studied together.
Note that the abundances of $\ceight$\ are very small:
$Y(\ceight)= 10^{-36}$ at $T = 1802$ K.
The isomeric transition from linear to
ringed $\ceight$, for which we estimate a lifetime $\tau = 10$ s,
provides the nucleations for graphite growth.

We call attention to several conclusions to be drawn from Table \ref{tab:flows}.
Firstly, the strongest flows by a wide margin involving $\csix$\ are
\cfivectocsix,
which is very fast ($k = 3 \times 10^{-10} {\rm cm}^3 {\rm s}^{-1}$)
because vibrational excitation of $\csix$
obviates the need for creating a photon \citep{1999Sci...283.1290C},
and its inverse reaction which ejects a C atom. Furthermore, those two flows
are equal to each other to three significant figures, illustrating the near 
steady
state of the abundance pattern. Secondly, thermal dissociation of $\csix$\ is
very much faster than its oxidation, showing the small effect of oxidation
on the abundance pattern. The destroying flows from $\csix$\ to $\cfive$
stand in the approximate ratio
${\rm C}_6(\gamma,{\rm C})/{\rm C_6}({\rm O},{\rm CO}) = 5 \times 10^5$
at $T = 2031$ K and $2 \times 10^4$ at $T = 1802$ K.
Dissociation by thermally excited vibrations dominates oxidation. What we here
label ($\gamma,{\rm C}$) is actually radiationless. Note that oxidation of
$\cenn$ molecules is not faster than C-ejection reactions. This unusual
situation occurs because the C-capture reactions proceed by exciting 
vibrations of the $\csix$ molecule. That residual vibrational energy makes
capture reactions fast rather than slow, but also makes the inverse C-ejection
reactions caused by thermally excited vibrations to be much faster. Detailed
balance gets the ratio right. The smaller value for that ratio at $T=1802$ K
occurs because oxidation maintains its effectiveness as $T$ drops but
thermal disruption does not, being much more temperature sensitive.
For this reason the abundance pattern is flatter and $\csix$\ is approximately
3000 times more abundant at $T=1802$ K than at $T=2031$ K.
It is for that reason that destruction flows in Table \ref{tab:flows}
for $\csix$\ are
larger at the smaller temperature.
The steady state shifts to new ratios as $T$ falls.
Thirdly, calculation of the rates will be illustrated for sake of clarity
by the flow
\csixotococfive.
From Fig. \ref{fig:cn}
one sees that at $T = 2031$ K the abundance $Y(\csix) = 10^{-33.3}$.
The value of $Y(O) = 10^{10} cm^{-3} / 1.1 \times 10^{10} cm^{-3} = 0.909$,
so the flow per atom per
second (e.g., Eq. \ref{eq:dycodt}) is
$dY/dt= Y(\csix)Y(O)n(2031K)k(\csix + {\rm O})$.
Expansion from $T=6000$ K to $T=2031$ K has diluted the total number
density $n$ to $1.1 \times 10^{10}{\rm cm}^{-3}
(2031 {\rm K}/6000 {\rm K})^3 = 4.27 \times 10^8{\rm cm}^{-3}$.
The fast oxidation reaction rate factor is
$k(\csix + {\rm O}) = 3 \times 10^{-10} {\rm cm}^3{\rm s}^{-1}$.
Gathering factors yields approximately $4 \times 10^{-35}$ per atom
per second
in good approximation to the Table \ref{tab:flows}
flow entry $3.61 \times 10^{-35}$ per atom per second.

Our calculations have shown clearly that producing $\ctwo$\ molecules near
5000 K via the CO + C reaction \citep{2009ApJ...703..642C} is not a viable
prospect for condensation of carbon dust via linear carbon chains.
The abundance of CO is far too small for it to seed high-T $\ctwo$\ production.
The abundance of $\ceight$
rings grows much later (Fig. \ref{fig:yi}) near 2000K, as
\citet{1999Sci...283.1290C,2001ApJ...562..480C} had found.
These rings grew from $\ctwo$\ created by simple carbon association,
\twoctoctwo.
Furthermore, the abundance of $\ctwo$\ is very small near 5000 K
(Fig. \ref{fig:ncoc2}),
primarily because its thermal dissociation rate is much too fast for
it to have significant abundance at that high $T$.
What little $\ctwo$\ is made at 5000 K is immediately dissociated by
thermal photos. Therefore its small steady-state abundance is inadequate
for building abundances of $\cthree$\ and beyond. 

\section{Contrast with \citet{2009ApJ...703..642C,2010ApJ...713....1C}}

The large numerical differences between the results of
\citet{1999Sci...283.1290C,2001ApJ...562..480C}
and those of \citet{2009ApJ...703..642C,2010ApJ...713....1C}
seem to be characterized by these differences:
\begin{enumerate}

\item Instead of growing large CO abundance near $Y(CO)=0.1$ at $T=5500$ K
as in Fig. 11 of \citet{2009ApJ...703..642C}, we find that $Y(CO)$
builds to a maximum of only $10^{-3}$, which it achieves only slowly
(Fig. \ref{fig:ncoc2}), reaching that maximum only at $T=3500$ K rather
than 5500 K.

\item We find a maximum number density $n_{\ctwo} = 0.1$ cm$^{-3}$,
only about $10^{-7}$ of
$n_{CO}$. We find $\ctwo$\ to be rare at high temperature because its
dissociation rate by thermal photons is very much faster than for CO
owing to the smaller binding energy (6.3eV) of $\ctwo$. This $n_{\ctwo}$
maximum differs greatly from results in \citet{2009ApJ...703..642C},
where their Fig. 11 shows the maximum atom fraction of $n_{\ctwo}$ to be
near $10^{-5}$, equal to about $10^5$ cm$^{-3}$, almost 1\% of
$n_{CO}$ at that time. Our $\ctwo$/CO abundance ratio is,
in other words, only $10^{-5}$ of the same ratio calculated by
\citet{2009ApJ...703..642C}.

\end{enumerate}
We reason that these big differences can be understood if
\citet{2009ApJ...703..642C} had inadvertantly omitted the thermal
dissociation rates of small carbon molecules. Believing that to be the
cause for the discrepancy, we tested that hypothesis by performing our
own trial calculation involving only C, O, CO, $\ctwo$, and $\cthree$,
as they had done,
and omitted the $\tau_\gamma$ destruction terms and the oxidation of $\cthree$.
Figure \ref{fig:no_gamma} displays the result.
The rapid rise of $Y(CO)$ to about $10^{-2}$ at high temperature is very 
similar to Fig. 11 of \citet{2009ApJ...703..642C}.
Similar also is $Y(\ctwo)$,
which grows quickly (Fig. \ref{fig:no_gamma}) to $2 \times 10^{-4}$ of $Y(CO)$,
whereas our Fig. \ref{fig:ncoc2} shows the true value of $Y(CO)$ to be
only $10^{-8}$ at $T=5500$ K and, considerably later,
$Y(\ctwo)/Y(CO) = 10^{-7}$ at their maxima. 

With photodissociation turned off, the destruction of $\ctwo$\ occurs primarily
by oxidation, \ctowotococ, whereas production of $\ctwo$ is by
\coctoctwo, the reaction we study in this work.
Those reactions strive to balance, which if achieved would
establish a steady state
ratio $Y(CO)/Y(\ctwo) = 3.5 \times 10^4$. Fig. \ref{fig:no_gamma}
and their Fig. 11 do show
approximately that value, but $Y(\ctwo)$ declines faster than $Y(CO)$
because of declining production of $\ctwo$\ as T declines.

We could not expect detailed agreement with \citet{2009ApJ...703..642C},
even if our hypothesis for the
cause of the discrepancy is correct. Our calculation used
O/C = 10 whereas they used O/C = 3 for the zone of their SNII model.
The temperature profiles also differ, we using $T=3800 {\rm K}/(t/100d)$
and they $T=18500 {\rm K}/(t/100d)^{1.8}$.
We believe that their $T$ is too hot owing to omission of CO cooling
(\citealp{1996ApJ...471..480L}, Fig. 5),
which we tried to accommodate roughly by
the choice $T=3800$ K at $t=100 {\rm d}$,
which is about 50\% of the temperatures
published within models that do omit CO cooling. And owing to the
factor $t^{-1.8}$, their $T$ falls through a specified temperature drop
(say, 5000 K to 3000 K) more quickly than does our parameterization.
Nonetheless, any $T$ profile declines though 5000 K and reaches 3000 K
somewhat later, so basically similar abundance results are expected. 

Such detailed differences are small in comparison with inclusion of
thermal photodissociation. The similarity of our Fig. \ref{fig:no_gamma}
to Fig. 11 of \citet{2009ApJ...703..642C},
and the huge differences of these figures from those of our network
amounts in our minds to a resolution of the discrepancy. 

\acknowledgments

T. Y. gratefully acknowledges the support of
a NASA Earth and Space Science Fellowship.
This work was also supported by NASA grant NNX10AH78G.

\clearpage

\begin{figure}
\plotone{ncoeq}
\caption{
Comparison of the density of CO molecules calculated by the numerical
network (curve) with the expectation $\ncogeq = n_C n_O k_{CO} \tau_\gamma$
for an abundance
$n_{CO}$ in thermal equilibrium (dots) with photons.
Above T=4000K the CO abundance is seen to be in an accurate thermal
equilibrium, but at lower temperature $n_{CO}$ is increasingly smaller than
thermal equilibrium would require. Instead of thermal photons,
the destruction of CO becomes increasingly dominated by
Compton electrons for $T < 3500$ K.
One see in Fig. \ref{fig:nco} that $n_{CO}$max occurs near 3500K,
which occurs near $10^7$ s (see Table \ref{tab:quantities}).} 
\label{fig:nco}
\end{figure}

\clearpage

\begin{figure}
\plotone{co_flows}
\caption{
Flows into and out of CO illustrate the near steady state between creation
and destruction and the transition region near 3500K between the two modes
of destruction of CO.
The destruction flow $n_{CO} n_C k_{CO+C}$ is only $3.27 \times
10^{-16}$ atom$^{-1}$s$^{-1}$, too small to be visible in Fig. \ref{fig:flows}. 
} \label{fig:flows}
\end{figure}

\clearpage

\begin{figure}
\plotone{nco}
\caption{
Number densities of CO and of $\ctwo$\ as functions of temperature.
The maximum density of $\ctwo$\ is 0.1 cm$^{-3}$.
Both abundances grow much more slowly than in the calculation by
\citet{2009ApJ...703..642C}.} \label{fig:ncoc2}
\end{figure}

\clearpage

\begin{figure}
\plotone{Yi}
\caption{
The abundance (number per atom) of linear chains $\cenn$\ and of ringed isomer
$\ceightr$. We have scaled each $Y(C_n)$ by the factor stated in the box.}
\label{fig:yi}
\end{figure}

\clearpage

\begin{figure}
\plotone{cn}
\caption{
Shape of $Y(C_n)$ vs. n for $\cthree$\ to $\ceight$
at two different temperatures
near 2000K. These shapes can be understood by the relative magnitudes of
flows into and out of $\cenn$.
}
\label{fig:cn}
\end{figure}

\clearpage

\begin{figure}
\plotone{no_gamma}
\caption{
Abundances for network containing only C, O, CO, $\ctwo$, and $\cthree$
when photodissociation of molecules is omitted. Rapid rise of $Y(CO)$
to $10^{-2}$ near 5500 K and the ratio $Y(CO)/Y(\ctwo)$ near $10^4$ are
similar to Fig. 11 of \citet{2009ApJ...703..642C}.
}
\label{fig:no_gamma}
\end{figure}

\clearpage

\begin{table}
\begin{center}
\caption{Values of $\tau_\gamma$(CO) and related quantities at
selected time $t$ and temperature $T$}
\label{tab:quantities}
\begin{tabular}{ccccccc}
\tableline\tableline
$t$ (s) & $\left[t / t(6000 {\rm K}) \right]^{-3}$ & $T$ (K) & $\tau_\gamma$ (s) & $k_{CO}$ (cm$^{-3}$ s$^{-1}$) & $\ncogeq$ (cm$^{-3}$) \\
\tableline
$5.47 \times 10^6$ &
  1 &
  6000 &
  $4.22 \times 10^{-2}$ &
  $3.13 \times 10^{-17}$ &
  $1.32 \times 10^1$ \\
$5.97 \times 10^6$ &
  0.769 &
  5500 &
  $3.54 \times 10^{-1}$ &
  $2.99 \times 10^{-17}$ &
  $6.27 \times 10^1$ \\
$6.57 \times 10^6$ &
  0.578 &
  5000 &
  $4.46 \times 10^{0}$ &
  $2.83 \times 10^{-17}$ &
  $4.24 \times 10^2$ \\
$7.30 \times 10^6$ &
  0.422 &
  4500 &
  $9.70 \times 10^{1}$ &
  $2.67 \times 10^{-17}$ &
  $4.61 \times 10^3$ \\
$8.21 \times 10^6$ &
  0.296 &
  4000 &
  $4.43 \times 10^{3}$ &
  $2.48 \times 10^{-17}$ &
  $9.67 \times 10^4$ \\
$9.38 \times 10^6$ &
  0.198 &
  3500 &
  $5.86 \times 10^{5}$ &
  $2.28 \times 10^{-17}$ &
  $5.26 \times 10^6$ \\
$1.09 \times 10^7$ &
  0.125 &
  3000 &
  $3.77 \times 10^{8}$ &
  $2.05 \times 10^{-17}$ &
  $1.21 \times 10^9$ \\
\tableline
\end{tabular}
\end{center}
\end{table}

\clearpage

\begin{table}
\begin{center}
\caption{Reaction Flows involving $\csix$\ at two temperatures.}
\label{tab:flows}
\begin{tabular}{ccc}
\tableline\tableline
& \multicolumn{2}{c}{Flow (${\rm atom}^{-1} {\rm s}^{-1}$)} \\
\cline{2-3}
Reaction & $T = 2031 {\rm K}$ & $T = 1802 {\rm K}$ \\
$\cfive({\rm C},\gamma)\csix$ &
  $1.67 \times 10^{-29}$ &
  $4.79 \times 10^{-28}$\\
$\csix(\gamma,{\rm C})\cfive$ &
  $1.67 \times 10^{-29}$ &
  $4.79 \times 10^{-28}$\\
$\csix({\rm C},\gamma)\cseven$ &
  $2.41 \times 10^{-39}$ &
  $2.12 \times 10^{-32}$\\
$\cseven(\gamma,{\rm C})\csix$ &
  $1.26 \times 10^{-40}$ &
  $7.27 \times 10^{-40}$\\
$\csix({\rm O},{\rm CO})\cfive$ &
  $3.61 \times 10^{-31}$ &
  $2.90 \times 10^{-32}$\\
\tableline
\end{tabular}
\end{center}
\end{table}

\begin{thebibliography}{14}
\expandafter\ifx\csname natexlab\endcsname\relax\def\natexlab#1{#1}\fi

\bibitem[{{Cherchneff} \& {Dwek}(2009)}]{2009ApJ...703..642C}
{Cherchneff}, I. \& {Dwek}, E. 2009, \apjl, 703, 642

\bibitem[{{Cherchneff} \& {Dwek}(2010)}]{2010ApJ...713....1C}
---. 2010, \apj, 713, 1

\bibitem[{{Clayton}(2011)}]{2011NewAR..55..155C}
{Clayton}, D.~D. 2011, \nar, 55, 155

\bibitem[{{Clayton}(2013)}]{2013ApJ...762....5C}
---. 2013, \apj, 762, 5

\bibitem[{{Clayton} {et~al.}(2001){Clayton}, {Deneault}, \&
  {Meyer}}]{2001ApJ...562..480C}
{Clayton}, D.~D., {Deneault}, E.~A.-N., \& {Meyer}, B.~S. 2001, \apj, 562, 480

\bibitem[{{Clayton} {et~al.}(1999){Clayton}, {Liu}, \&
  {Dalgarno}}]{1999Sci...283.1290C}
{Clayton}, D.~D., {Liu}, W., \& {Dalgarno}, A. 1999, Science, 283, 1290

\bibitem[{{Clayton} \& {The}(1991)}]{1991ApJ...375..221C}
{Clayton}, D.~D. \& {The}, L.-S. 1991, \apj, 375, 221

\bibitem[{{Deneault} {et~al.}(2003){Deneault}, {Clayton}, \&
  {Heger}}]{2003ApJ...594..312D}
{Deneault}, E.~A.-N., {Clayton}, D.~D., \& {Heger}, A. 2003, \apj, 594, 312

\bibitem[{{Deneault} {et~al.}(2006){Deneault}, {Clayton}, \&
  {Meyer}}]{2006ApJ...638..234D}
{Deneault}, E.~A.-N., {Clayton}, D.~D., \& {Meyer}, B.~S. 2006, \apj, 638, 234

\bibitem[{{Klots}(1968)}]{1968...Klots}
{Klots}, C.~E. 1968, in Fundamental Processes in Radiation Chemistry, ed.
  P.~{Ausloos} (New York: Interscience)

\bibitem[{{Lepp} {et~al.}(1990){Lepp}, {Dalgarno}, \&
  {McCray}}]{1990ApJ...358..262L}
{Lepp}, S., {Dalgarno}, A., \& {McCray}, R. 1990, \apj, 358, 262

\bibitem[{{Liu} \& {Dalgarno}(1996)}]{1996ApJ...471..480L}
{Liu}, W. \& {Dalgarno}, A. 1996, \apj, 471, 480

\bibitem[{{Liu} \& {Victor}(1994)}]{1994ApJ...435..909L}
{Liu}, W. \& {Victor}, G.~A. 1994, \apj, 435, 909

\bibitem[{{Meyer} \& {Adams}(2007)}]{2007M&PSA..42.5215M}
{Meyer}, B.~S. \& {Adams}, D.~C. 2007, 70th Annual Meteoritical Society
  Meeting, held in August 13-17, 2007, Tucson, Arizona.~Meteoritics and
  Planetary Science Supplement, Vol.~42, p.5215, 42, 5215

\bibitem[{Takai {et~al.}(1990)Takai, Lee, Halicioglu, \&
  Tiller}]{doi:10.1021/j100374a025}
Takai, T., Lee, C., Halicioglu, T., \& Tiller, W.~A. 1990, The Journal of
  Physical Chemistry, 94, 4480

\end{thebibliography}

\end{document}
