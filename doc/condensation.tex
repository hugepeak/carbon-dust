\documentclass{article}

\begin{document}

\section{Abundance Change}

For a reaction:
\[
1 + 2 \longleftrightarrow 3 + \gamma
\]
we can write the number density change of species 1 as:
\begin{equation}
\frac{{\rm d}n_1}{{\rm d}t} = 
  -k \, n_1 n_2 + \lambda_\gamma n_3
\label{eq:number_density}
\end{equation}
where $k$ is the reaction rate in the unit of $cm^3s^{-1}$ and 
$\lambda_\gamma$ is the photodissociation rate of species 3 with the unit of
$s^{-1}$.

In a expanding or a collapsing system the number densities change with volume. 
We'd like to study a quantity that does not depend on volume directly so that
we use the fraction of each species in number instead of number density:
\[
Y_i \equiv n_i / n_{total}
\]
where $Y_i$ is called the abundance of species i, $n_i$ is the number density
of species i and $n_{total}$ is the number density of the unit particles, here
the number density of total atoms. So $Y_i$ here indicates the number
of species i per atom. And
\[
n_{total} = n_{atom} = n_{baryon} \times Y_{atom} 
\]
where $n_{baryon} = \rho N_A$ is the baryon number density and $Y_{atom}$ is
the atom fraction, i.e. number of atoms per baryon. Take pure $^{12}$C for 
example, $Y_{atom} = 1 / 12$. So Eq.\ref{eq:number_density} can be rewrite as
\begin{equation}
\frac{{\rm d}Y_1}{{\rm d}t} = 
  -k n_{atom} Y_1 Y_2 + \lambda_\gamma Y_3 =
  -k \rho N_A Y_{atom} Y_1 Y_2 + \lambda_\gamma Y_3
\label{eq:abundance}
\end{equation}
where $\rho$ is the mass density and $N_A$ is the Avogadro number. Then what
we need to study is how the abundances of each species change with time.

\section{Condensation Effective Rates}

For a single kind of atom, we study the effective condensation rate here.
The condensation here means one atom attaches to an existing multi-atom
particle:
\begin{equation}
P_1 + P_i \longleftrightarrow P_{i+1} + \gamma
\label{eq:reaction}
\end{equation} 
where $P_i$ denotes a particle with $i$ atoms in it.

Let's assume that the condensation rate is proportional to the surface area
(or cross section) of the larger particle, which goes like $n^{2/3}$, i.e.
\[
k \propto A \propto n^{2/3}
\]
where $k$ is the rate, $A$ is the surface area and $n$ is the number of atoms
in the particle.

To make the calculation of a large atom number particle possible, we introduce
the ``bin" after a continuous size increment. In a bin we neglect the details 
of the single particles and focus on the
two ends of the bin. For example if we start the bin from size $a$ and the bin 
size is 10, the particles we care about after size $a$
are just $P_a$, $P_{10a}$, $P_{100a}$, etc. 
And we just calculate the effective reaction rates between these particles.

From graph theory we know that for a series of vertices 1, 2, ..., N, the
effective rate between 1 and N is like:
\[
k_{1,N}^{eff} \Delta t = \frac 
  { k_{1,2}k_{2,3}...k_{N-1,N}\Delta t^{N-1} }
  { 
    k_{2,3}k_{3,4}...k_{N-1,N}\Delta t^{N-2} +
    k_{1,2}k_{3,4}...k_{N-1,N}\Delta t^{N-2} +
    ...
    k_{1,2}...k_{N-2,N-1}\Delta t^{N-2}
  }
\]

\[
k_{1,N}^{eff} = \frac {1}
  { 
    \frac{1}{k_{1,2}} +
    \frac{1}{k_{2,3}} +
    ...
    \frac{1}{k_{N-1,N}}
  }
\]

Back to the problem we are looking at(Eq.\ref{eq:reaction}), we can write
\[
k_{i,i+1} \equiv k_i = k_1 i^{2/3}
\]
If we start from $p_a$, for the bin size of $b$ we can write the effective
reaction as:
\[
P_a + a\times (b-1) P_1 \to P_{a\times b}
\]
And the effective rate is
\[
k^{eff} = \frac {1}
  {
    \frac {1} {k_{a}} +
    \frac {1} {k_{a+1}} +
    ...
    \frac {1} {k_{a\times b}}
  }
\]
\[
\frac {1} {k_{eff}} =
  \sum_{i=a}^{a\times b} \frac {1} {k_i} =
  \frac{1}{k_1} \sum_{i=a}^{a\times b} \frac {1} {i^{2/3}} \approx
  \frac{1}{k_1} \int_a^{a\times b} x^{-2/3} {\rm d} x =
  \frac{1}{k_1} \cdot 3 \cdot [(a\times b)^{1/3} - a^{1/3}]
\]
\[
k^{eff} = \frac{k_1}{3} 
  \frac{1}{ (a\times b)^{1/3} - a^{1/3} }
\]

If we denote the atom number at the beginning of a bin as $N_{begin}$ and
the one at the end as $N_{end}$, then we can write the effective rate as:
\begin{equation}
k^{eff} = \frac{k_1}{3} 
  \frac{1}{ N_{end}^{1/3} - N_{begin}^{1/3} }
\label{eq:rate}
\end{equation}

We estimate $k_1$ as two single atoms collide together with thermal velocity
and the transverse section.
\begin{equation}
k_1 = \: <\! \sigma_1 v\! > \: = \pi R^2 \sqrt{3k_BT/m}
\label{eq:k1}
\end{equation}
where $\sigma_1$ is the cross section for a single atom and $R$ is the radius
of a single atom, $v$ is the thermal velocity which is taken to be the average
value in 3 dimensions, $k_B$ is the Boltzmann constant, $T$ is the temperature
and $m$ is the mass of a single atom.

\section{Atom Conservation}

The abundance change rates of $P_{begin}$, $P_{end}$ and $P_1$ due to 
reaction $P_{begin} + X P_1 \longleftrightarrow P_{end} + \gamma$ are
\[
\frac{{\rm d}Y_{begin}}{{\rm d}t} =
  -k^{eff} n_{atom} Y_{begin} Y_1 + \lambda_\gamma^{eff} Y_{end}  
\]
\[
\frac{{\rm d}Y_{end}}{{\rm d}t} =
  k^{eff} n_{atom} Y_{begin} Y_1 - \lambda_\gamma^{eff} Y_{end}  
\]
\[
\frac{{\rm d}Y_1}{{\rm d}t} =
  X( -k^{eff} n_{atom} Y_{begin} Y_1 + \lambda_\gamma^{eff} Y_{end} )
\]
where $\lambda_\gamma^{eff}$ is the effective photodissociation rate which
will be discussed later and $X$ is the factor to be determined that ensure
the total atom number conserved. Atom conservation requires
\[
N_{begin} \times \frac{{\rm d}Y_{begin}}{{\rm d}t} +
N_{end} \times \frac{{\rm d}Y_{end}}{{\rm d}t} +
\frac{{\rm d}Y_1}{{\rm d}t} = 0.
\] 
So we have
\[
X = N_{end} - N_{begin}
\]

\section{Network Matrix}
Let's look at the reaction between the last species in the continuous network
and the first bin, and neglect the photodissociation here:
\[
P_{last} + XP_1 \to P_{b1}
\]
The abundance change rates are then (using $n_a$ to denote $n_{atom}$)
\[
\frac{{\rm d}Y_{last}}{{\rm d}t} =
  -k^{eff}_{last} \, n_a Y_{last} Y_1
\]
\[
\frac{{\rm d}Y_{b1}}{{\rm d}t} =
  k^{eff}_{last} \, n_a Y_{last} Y_1
\]
\[
\frac{{\rm d}Y_1}{{\rm d}t} =
  ( -k^{eff}_{last} \, n_a Y_{last} Y_1 )(N_{b1} - N_{last})
\]
Then the changes of the root-finding functions are (how to call them?)
\[
f_{last} \: +\!\!= \: k^{eff}_{last} \, n_a Y_{last} Y_1
\]
\[
f_{1} \: +\!\!= \: (k^{eff}_{last} \, n_a Y_{last} Y_1) (N_{b1} - N_{last})
\]
To find the abundance changes in the continuous network, we need to solve the
matrix
\[
AX=B
\]
The matrix element changes due to this reaction are
\[
A_{last,last} \: +\!\!= \: k^{eff}_{last} \, n_a Y_1
\]
\[
A_{last,1} \: +\!\!= \: k^{eff}_{last} \, n_a Y_{last}
\]
\[
A_{1,last} \: +\!\!= \: (k^{eff}_{last} \, n_a Y_1)(N_{b1} - N_{last})
\]
\[
A_{1,1} \: +\!\!= \: (k^{eff}_{last} \, n_a Y_{last})(N_{b1} - N_{last})
\]
The right-hand-side vector changes are
\[
B_{last} \: +\!\!= \: -f_{last} = -k^{eff}_{last} \, n_a Y_{last} Y_1 
\]
\[
B_1 \:+\!\!=\: -f_1 = -(k^{eff}_{last}\,n_a Y_{last} Y_1)(N_{b1}-N_{last})
\]

Now let's look at the reactions between bins. $P_{bi} + XP_1 \to P_{b(i+1)}$,
where $i$ goes from 1 to $N-1$ with $N$ as the number of bins.
\[
\left( \frac{{\rm d}Y_1}{{\rm d}t}\right) _i =
  ( -k^{eff}_{bi} \, n_a Y_{bi} Y_1 )(N_{b(i+1)} - N_{bi})
\]
\[
(f_{1})_i \: +\!\!= \: 
  (k^{eff}_{bi} \, n_a Y_{bi} Y_1) (N_{b(i+1)} - N_{bi})
\]
\[
(A_{1,1})_i \: +\!\!= \: 
  (k^{eff}_{bi} \, n_a Y_{bi}) (N_{b(i+1)} - N_{bi})
\]
\[
(B_1)_i \:+\!\!=\: -(f_1)_i = 
  -(k^{eff}_{bi} \, n_a Y_{bi} Y_1) (N_{b(i+1)} - N_{bi})
\]
We first solve the matrix equation for current $Y_{bi}$, and then compute
the abundance changes in bins using the new abundances in the network.
Then feed the new bin abundances back to the matrix. Iterate until converge. 

\section{Bin Abundances Change}

Now let's look at the abundance changes in bins for each reaction discussed
above. We are going to solve these changes implicitly. For $Y_{b1}$,
\[
\frac {Y_{b1}(t+\Delta t) - Y_{b1}(t)} {\Delta t} =
  k^{eff}_{last} \, n_a Y_{last}(t+\Delta t) Y_1(t+\Delta t) -
  k^{eff}_{b1} \, n_a Y_{b1}(t+\Delta t) Y_1(t+\Delta t)
\]
\[
Y_{b1}(t+\Delta t) = \frac
  {k^{eff}_{last} \, n_a Y_{last}(t+\Delta t) Y_1(t+\Delta t) \Delta t +
    Y_{b1}(t)}
  {1 + k^{eff}_{b1} \, n_a Y_1(t+\Delta t) \Delta t}
\]
Similarly for $Y_{bi}$ as $i$ goes from 2 to $N-1$:
\[
Y_{bi}(t+\Delta t) = \frac
  {k^{eff}_{b(i-1)} \, n_a Y_{b(i-1)}(t+\Delta t) Y_1(t+\Delta t) \Delta t +
    Y_{bi}(t)}
  {1 + k^{eff}_{bi} \, n_a Y_1(t+\Delta t) \Delta t}
\]
For $Y_{bN}$ we don't have any destructive terms for now,
\[
Y_{bN}(t+\Delta t) = 
  k^{eff}_{b(N-1)} \, n_a Y_{b(N-1)}(t+\Delta t) Y_1(t+\Delta t) \Delta t +
    Y_{bi}(t)
\]


\end{document}
