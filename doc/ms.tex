%%
%% Beginning of file 'sample.tex'
%%
%% Modified 2005 December 5
%%
%% This is a sample manuscript marked up using the
%% AASTeX v5.x LaTeX 2e macros.

%% The first piece of markup in an AASTeX v5.x document
%% is the \documentclass command. LaTeX will ignore
%% any data that comes before this command.

%% The command below calls the preprint style
%% which will produce a one-column, single-spaced document.
%% Examples of commands for other substyles follow. Use
%% whichever is most appropriate for your purposes.
%%
%%\documentclass[12pt,preprint]{aastex}

%% manuscript produces a one-column, double-spaced document:

\documentclass[manuscript]{aastex}

%% preprint2 produces a double-column, single-spaced document:

%% \documentclass[preprint2]{aastex}

%% Sometimes a paper's abstract is too long to fit on the
%% title page in preprint2 mode. When that is the case,
%% use the longabstract style option.

%% \documentclass[preprint2,longabstract]{aastex}

%% If you want to create your own macros, you can do so
%% using \newcommand. Your macros should appear before
%% the \begin{document} command.
%%
%% If you are submitting to a journal that translates manuscripts
%% into SGML, you need to follow certain guidelines when preparing
%% your macros. See the AASTeX v5.x Author Guide
%% for information.

\newcommand{\myemail}{skywalker@galaxy.far.far.away}
\newcommand{\ctwo}{${\rm C}_2$}
\newcommand{\cthree}{${\rm C}_3$}
\newcommand{\cfour}{${\rm C}_4$}
\newcommand{\cfive}{${\rm C}_5$}
\newcommand{\csix}{${\rm C}_6$}
\newcommand{\cseven}{${\rm C}_7$}
\newcommand{\ceight}{${\rm C}_8$}
\newcommand{\ceightr}{${\rm C}_8^r$}
\newcommand{\nar}{New Astron. Rev.}

%% You can insert a short comment on the title page using the command below.

%\slugcomment{Not to appear in Nonlearned J., 45.}

%% If you wish, you may supply running head information, although
%% this information may be modified by the editorial offices.
%% The left head contains a list of authors,
%% usually a maximum of three (otherwise use et al.).  The right
%% head is a modified title of up to roughly 44 characters.
%% Running heads will not print in the manuscript style.

\shorttitle{Formation of ${\rm C}_n$ Molecules}
\shortauthors{Yu et al.}

%% This is the end of the preamble.  Indicate the beginning of the
%% paper itself with \begin{document}.

\begin{document}

%% LaTeX will automatically break titles if they run longer than
%% one line. However, you may use \\ to force a line break if
%% you desire.

\title{Formation of ${\rm C}_n$ Molecules in Oxygen-rich Interiors of Type II
Supernovae}

%% Use \author, \affil, and the \and command to format
%% author and affiliation information.
%% Note that \email has replaced the old \authoremail command
%% from AASTeX v4.0. You can use \email to mark an email address
%% anywhere in the paper, not just in the front matter.
%% As in the title, use \\ to force line breaks.

\author{Tianhong Yu, Donald D. Clayton, and Bradley S. Meyer}
\affil{Department of Physics and Astronomy, Clemson Universit, Clemson, SC 29634-0978}

%% Notice that each of these authors has alternate affiliations, which
%% are identified by the \altaffilmark after each name.  Specify alternate
%% affiliation information with \altaffiltext, with one command per each
%% affiliation.

%% Mark off your abstract in the ``abstract'' environment. In the manuscript
%% style, abstract will output a Received/Accepted line after the
%% title and affiliation information. No date will appear since the author
%% does not have this information. The dates will be filled in by the
%% editorial office after submission.

\begin{abstract}
Two reaction-rate-based kinetic models for condensation of carbon dust via the
growth of precursor linear carbon chains are currently under debate. The first
involved forming \ctwo\ molecules via radiative association of free C
atoms; the second forms \ctwo\ molecules by the endoergic reaction
${\rm CO} + {\rm C} \to {\rm C}_2 + O$. Both are
followed by C captures until the linear chain eventually makes an isomeric
transition to ringed carbon on which rapid growth of graphite may occur.
These two approaches give vastly different results. Because of the high
importance of condensable carbon for current problems in astronomy,
we study these competing claims with an intentionally limited reaction-rate
network that shows clearly that initiation by
${\rm C} + {\rm C} \to {\rm CO} + \gamma$  is the dominant
pathway to carbon rings. We propose an explanation for why the second pathway
is not nearly as effective as its proponents calculated it to be.
\end{abstract}

%% Keywords should appear after the \end{abstract} command. The uncommented
%% example has been keyed in ApJ style. See the instructions to authors
%% for the journal to which you are submitting your paper to determine
%% what keyword punctuation is appropriate.

\keywords{globular clusters: general --- globular clusters: individual(NGC 6397,
NGC 6624, NGC 7078, Terzan 8}

\section{Introduction}

The problem historically of condensing carbon SUNOCONs in the interiors of
expanding Type II supernovae is that most of the synthesized carbon is bathed
in more abundant oxygen and was long presumed to be oxidized by it. This
changed when \citet{1999Sci...283.1290C} introduced a kinetic theory by
which carbon can condense thermally in gas having more abundant oxygen.
That theory is noteworthy in having every chemical reaction from the initial
O $>$ C gas to macroscopic graphite included in a kinetic network, even though
most of the cross sections for those reactions and the rates of isomeric
transitions from linear carbon chains to carbon rings are not accurately known.
Their paper provided an existence proof that carbon can condense in O-rich gas.
Its calculations showed that rare carbon-ring molecules were not prevented from
formation despite rapid oxidation of their molecular precursors, although
oxidation does greatly reduce the abundance of those Cn chains. That initial
theory has been amplified during the past decade by several studies
\citep{2001ApJ...562..480C,2003ApJ...594..312D,2006ApJ...638..234D,
2009ApJ...703..642C,2010ApJ...713....1C,2011NewAR..55..155C,
2013ApJ...762....5C}.

In basic agreement with this chemical model, \citet{2009ApJ...703..642C} raised
questions about the most effective way of producing \ctwo\ molecules,
which is the
first step to linear carbon chains. Using a large network of chemical
reactions, \citet{2009ApJ...703..642C} discovered that the neutral-neutral
reaction $CO + C \to C_2 + O$
has a reasonably large cross section at temperatures near
5000K despite being endothermic by 4.8eV. Although the CO target abundance for
production of \ctwo\ in this manner must be calculated, they stated that at
$T=5000$K the rate factor
$\langle \sigma v \rangle = k_{CO,C} = 8.6 \times 10^{-14}$ cm$^3$ s$^{-1}$
created \ctwo\ much more
rapidly than the slow radiative reaction $C + C \to C_2 + \gamma$.
Sensing its significance,
they claimed in their list of conclusions: ``2. A new pathway to the formation
of carbon chains is active in the O-rich mass zone of the unmixed ejecta and
is identified as the CO conversion to \ctwo\ via collisions with C.''  They were
unable to evaluate its full quantitative consequences because their reaction
network terminated at \cthree\
and did not include the formation of linear carbon
chains Cn and their isomerization to carbon rings. Doing so is one goal of our
study. But the overriding goal is to point out large abundance differences
between the computations of these two groups and to resolve them if possible. 

Several important consequences for astronomy depend on careful
analysis of this claim that the production of \ctwo\ owing to the reaction
$C+C \to C_2 + \gamma$, as \citet{1999Sci...283.1290C,2001ApJ...562..480C}
had done, severely
underestimates the abundances of linear carbon chains because much more \ctwo\
is made earlier \citep{2010ApJ...713....1C} near T=5000K. Testing this
competition is essential for understanding of dust condensation within
supernovae. Dust created by supernova expansions is currently under intense
study owing to three types of astronomical observations: (1) of dust observed
in single supernova remnants; (2) of dust observed in early low-metallicity
galaxies; (3) of supernova-condensed carbon dust (SUNOCONs) extracted from
meteorites. Each of these topics depends sensitively on how much carbon,
both numbers and sizes, is able to condense in cooling SNII interiors.
Therefore we study this competition carefully.

Crucial to this task is the lifetime τCO of CO molecules. Their thermal
dissociation lifetime τγ is calculated from detailed balance with the
radiative association reaction, as in Section 2.1 of
\citet{2001ApJ...562..480C}.
Because of the large 11.1eV binding energy of the CO molecule,
$\tau_\gamma$ is very
temperature sensitive.  For readers numerical ease we will tabulate
$\tau_\gamma$ at
selected key temperatures in Table \ref{tab:quantities} to follow.
The flux and spectrum of newly
injected Compton electrons \citep{1991ApJ...375..221C} yields the lifetime
$\tau_e$ of CO
molecules against inelastic scattering dissociation by Compton electrons
caused by the 56Co radioactivity. In a gas of pure CO the mean energy per ion
pair is defined as the energy of primary electrons divided by the number of
pairs produced. Liu and Victor (1994) calculated the mean energy per ion pair
in pure CO gas and obtained the result ΔE=32.3 eV deposited per dissociated CO
pair, which agrees well with the measurement of 32.2 eV by
\citet{1968...Klots}.
So the efficiency of energetic electrons for dissociating CO seems well
established. A lifetime $\tau_{CO}$ near one week is typical in SN 1987A, but
would be longer in SNII synthesizing less $^{56}$Co and at times greater than
about 8 months.

\section{The Physical Model and Network}

Because our goal is to study the chemistry, we can take a very simple physical
model of the expansion; namely, temperature T=3800/(t/100d)=3.30x1010Ks/t(s),
where t is the time elapsed since core collapse and nO=1010cm-3 and
$n_C = 10^9$ cm$^{-3}$
at the starting time t=t(6000K) for our chemical network at
T=6000K. From our model choice for T(t) we get
$t(6000K) = 5.47 \times 10^6$s. At
subsequent times nO(t)=1010cm-3(t/t(6000))-3 owing to homologous expansion.
Let N be the number in a comoving, expanding, initially 1cm3 volume at 6000K
that held initially nO=1010cm-3 and nC=109cm-3. The only change of N during
expansion occurs through chemical reactions. N may be expressed as atom
fraction Y of the initial total number NO+NC= 1.10x1010 atoms. We
intentionally choose an O-rich interior having NO/NC=10, so that no carbon at
all would be able to condense if that interior were governed by chemical
equilibrium. 

We use an intentionally limited set of chemical species because our goal is
to study carbon chemistry: C, O, CO, \ctwo, \cthree, ... \ceight and
\ceightr, the ringed isomer of \ceight, to which we give a lifetime
$\tau_8^r = 10$ s against thermal isomeric transition
from linear \ceight to ringed \ceightr.
Those ringed molecules are the seeds for carbon
growth because their oxidation rates are much smaller than oxidation rates of
linear Cn whereas its C capture rates are fast. We take ringed \ceightr to be
indestructible, simply integrating their rate of production. Our strategy is
to compute the number \ceightr remaining after expansion from two differing
sources of \ctwo\ initiating the linear Cn chains. Those two source reactions
are
\begin{itemize}
\item $C + C \to C_2 + \gamma$ \citep{1999Sci...283.1290C}.
\item $CO + C \to C_2 + O$  \citep{2009ApJ...703..642C,2010ApJ...713....1C}.
\end{itemize}
We take our reaction rates from the rate tables in
\citet{2009ApJ...703..642C,2010ApJ...713....1C}.
The rate for the second reaction is indeed
much greater than that of the first reaction above T=3000K;
but it becomes increasingly the smaller of the two below T=2500K.
We include the thermal inverse reaction of every reaction,
which we calculate as in Eq.(3) of \citet{2001ApJ...562..480C}.
See Table 1 below for sample thermal dissociation rates for CO molecules.
We also include in our network dissociation of CO by Compton electrons.
CO is the only molecule for which thermal photodissociation is the dominant
destruction mechanism. For that rate we use
$\tau_e = 10^5$s $\exp( (t-10^6{\rm s}) / 111 {\rm d} )$,
not fit to any specific model but with plausibility for SN1987A.
It lengthens to $\tau_e = 10^6$ s near eight months.
Because of their helpful cataloging of reaction rate tables,
and in order to compare results without rate differences, we take the
rates as given in the tables of
\citet{2009ApJ...703..642C,2010ApJ...713....1C}.
As one example, we take the rate for $C + C \to C_2 + \gamma$
as the rate given by RA4 in Table 5 in \citet{2009ApJ...703..642C}, p. 649.

Our computational reaction network is {\em libnucnet}
\citep{2007M&PSA..42.5215M} modified to carry the chemical
rather than nuclear network.  This network code has been thoroughly tested
on a wide variety of reaction networks and problems.
We scrupulously tested our network answers by hand calculations capable of
exposing coding errors.

\section{The Function $n_{CO}(t)$}

The interior core of C and O resulting from completed He burning in massive
stars is a mix of C and O atoms having bulk C/O $<$ 1. Post explosive cooling of
such matter will attempt to associate C and O into CO molecules, with rate
coefficient $k_{CO}$=3.3x10-17cm3s-1 \citep{1990ApJ...358..262L}.
The reaction C+O → CO+ γ is one of the crucial reactions of chemical
astrophysics. Its rate $k_{CO}$ is intrinsically slow because quantum mechanics
requires not only rearrangement of electronic shells but also simultaneous
creation of a photon during the collision; nonetheless, the huge product
nCnO in the He-exhausted core ensures steady growth for the CO abundance
until it is reversed by radioactive dissociation. 

Evidently the abundance of CO within that zone during expansion attempts to
balance these creation and destruction effects, doing so exactly at the
time of maximum nCO. \citet{2013ApJ...762....5C}
has discussed the shape of the function
nCO(t). His Eq.(1) approximates the growth of nCO by its leading terms:
\begin{equation}
\frac{dn_{CO}}{dt} = n_{C(g)} n_O k_{CO} - \frac{n_{CO}}{\tau_{CO}} = 0
\label{eq:dncodt}
\end{equation}
at $t = t_{max}$.
The maximum abundance reached by CO is in cgs units
\begin{equation}
n_{CO}(t_max) = n_C n_O k_{CO} \tau_{CO} 
\label{eq:comax}
\end{equation}

This amount is equal to that formed during its last mean lifetime $\tau_{CO}$
against dissociation. If $\tau_{CO}$ is taken to be $\tau_\gamma$ because
the dissociation of CO is dominated by thermal photons,
one obtains the expression for the abundance $n_{CO}^{\gamma eq}$ in thermal
equilibrium: $n_{CO}^{\gamma eq} = n_C n_O k_{CO} \tau_\gamma$,
and is also entered in Table 1. 

Expressed instead in terms of number fractions $Y_{CO} = n_{CO} / n$
and $Y_C = n_C / n$ where $n$ is the number density of all atoms,
Eq. (\ref{eq:dncodt}) transforms to 
\begin{equation}
\frac{dY_{CO}}{dt} = Y_C Y_O n k_{CO} - \frac{Y_{CO}}{\tau_{CO}}	
\label{eq:dycodt}
\end{equation}
which is the form integrated by our network of coupled reactions.
The number density n is needed for the rate n(cm-3)$k_{CO}$(cm3s-1).
In our numerical example we take $n_O = 10^{10}$ cm$^{-3}$ and
$n_C = 10^9$ cm$^{-3}$ at t(6000K) as starting conditions for the
chemical network. Subsequent values are $n = 1.1 \times 10^{10}$ cm$^{-3}$
$[t(6000)/t]^3$, where the expansion factor reduces the initial number
density appropriately. Expansion factors are also in Table 1. Then
Eq. (\ref{eq:comax}) reads:
\begin{equation}
Y_{CO}(t_{max}) = Y_C Y_O n k_{CO} \tau_{CO}
\label{eq:ycomax}
\end{equation}

The lifetime of CO against dissociation is a composite of two physical
reactions. Letting $\tau_\gamma$ be the photodissociation lifetime owing
to thermal photons and $\tau_e$ be the dissociation lifetime owing to fast
Compton electrons, we have
\begin{equation}
\frac{1}{\tau_{CO}} = \frac{1}{\tau_\gamma} + \frac{1}{\tau_e}
\label{eq:tau_co}
\end{equation}
Which partial lifetime dominates the dissociation depends on the temperature.
The radioactive lifetime $\tau_e$ is taken to be
$\tau_e = 10^5\ {\rm s} \exp( ( t -10^6{\rm s}) / 111 {\rm d} )$,
but the thermal photodissociation lifetime $\tau_\gamma$
depends strongly on temperature
owing to the large binding energy of the CO molecule.
Table \ref{tab:quantities} displays a
short list of $\tau_\gamma$ at key temperatures as well as several related
quantities.

One sees from Table \ref{tab:quantities}
that $\tau_\gamma$ dominates Eq. (\ref{eq:tau_co}) for $T > 3500$ K,
whereas Compton
electron dissociation $\tau_e$
dominates below 3500 K. We start computation of our
chemical network at $T = 6000$ K,
so $n_{CO}(t)$ will initially be small and will
grow as temperature declines owing to the lengthening of $\tau_\gamma$ with
falling temperature. After $t=t_{max}$ the abundance of CO declines owing to
the destruction rate $1/\tau_e$ exceeding the creation rate in
Eq. (\ref{eq:dncodt})
while the gas cools to temperatures at which carbon chains can grow and
isomerize to rings \citep{1999Sci...283.1290C}. Such rings are taken to be
the nucleations upon which graphite grows.

Figure \ref{fig:nco}
compares our network calculation of the abundance of CO molecules
with the expectation of Eq.(2) when τγ dominates the destruction of CO.
Solid points display the equilibrium product nCOγeq calculated from the
factors shown in Table \ref{tab:quantities}
and from the number densities $n_C$ and $n_O$ after
their initial values at T=6000K have been reduced by the expansion factor
(t/t(6000K))-3. Tight agreement for $T > 3500$ K is immediately evident,
demonstrating that for $T > 3500$ K the abundance of CO is almost exactly in
thermal equilibrium. Below 3500K the abundance of CO becomes much less than
the expectation nCOγeq  of thermal equilibrium because the dissociation of CO
comes to be dominated by Compton electrons. The dashed vertical line at
T=3500K marks the approximate boundary between these two mechanisms for CO
dissociation. Notice carefully in Fig. \ref{fig:nco}
that nCO grows slowly, not reaching
its maximum until expansion has cooled to T=3500K.
At maximum nCO= 1.5x106cm-3, corresponding to YCO=0.7x10-3. The growth of nCO
shown in Fig. \ref{fig:nco} differs markedly from the results of
\citet{2009ApJ...703..642C}.
Their results (e.g. Fig. 11) show nCO climbing quickly near 6000K to a large
maximum number fraction near 0.1. That maximum would almost exhaust free
carbon. Our results in Fig. \ref{fig:nco}
so differ from theirs that the difference
must be resolved. \citet{2013ApJ...762....5C} has analyzed the expectation
of growth of nCO to a single maximum before declining, and our results are
in line with that expectation. 

These features are further detailed in Fig. \ref{fig:flows},
which shows the reaction
currents (reactions per second per atom) into and out of CO. There exists a
near steady state in that the production of CO is almost balanced by the
two flows destroying CO. The destruction flow $n_{CO}/\tau_\gamma$
almost balances the
association flow above 4000K except for a tiny excess leading to the
slow growth of nCO evident in Fig. \ref{fig:nco}. In that temperature range the
production by C+O→ CO+γ is in equilibrium with the thermal radiation field,
as Fig. \ref{fig:nco} implied. The destruction flow nCO/τe almost balances the
creation flow below 3500K. The transition between destruction modes occurs
near 3500K. It is no coincidence that the maximum of nCO occurs when
Compton electrons begin to dominate CO dissociation, because nCO would
continue growing as long as thermal photons dominate CO dissociation. 

\section{Abundances of $C_n$}

Figure \ref{fig:ncoc2} displays the number density $n_{C_2}$ along with
that of $n_{CO}$. At its maximum $n_{C_2} =0.1$ cm$^{-3}$ is only about
$10^{-7}$ of $n_{CO}$. The reason \ctwo\ is so rare is that its dissociation
rate by thermal photons is much faster than that for CO because the binding
energy of {\bf \ctwo\ CORRECT?} is
so much less than that for CO. This $n_{C_2}$
maximum differs greatly
from results in \citet{2009ApJ...703..642C}, where their Fig. 11 shows the
maximum atom fraction of $n_{C_2}$ to be near $10^{-5}$,
equal to about $10^5$ cm$^{-3}$, almost 1\% of $n_{CO}$ at that time.
And their maximum for \ctwo\ occurs near 7000 K, far too hot for \ctwo\ to
be abundant in the face of rapid dissociation. The huge differences between
these two computations require an explanation. We find it likely that
\cite{2009ApJ...703..642C} inadvertently omitted photodissociation by thermal
photons from their destruction rates.  


Figure \ref{fig:yi}
shows the abundances $Y_i$ of each species in our small network as a
function of time t-t(6000K) after T=6000K. From Table \ref{tab:quantities},
start time is $t(6000K) =5.47x10^6$ s. To display each $Y_i$ on a figure with
reasonable ordinate resolution, we have scaled each $C_n$ by the factor
stated in the box. Many features are noteworthy: (1) C and O are constant
because their small depletion is negligible on the scale shown;
(2) maxima in CO and \ctwo\ occur at almost exactly the same time
$t-t(6000K)= 4 \times 10^6$ s, as in Fig. \ref{fig:ncoc2};
(3) \cthree\ has very small abundance, about $3 \times 10^{-8}$ of \ctwo
at the \ctwo\ maximum, although much later \cthree\ slowly grows modestly
relative to much more abundant \ctwo; (4) the rise shapes of \cfour
through \ceight\ are very similar because they are linked by a near steady
state; (5) the ringed carbon \ceightr\ has similar abundance shape vs. time,
but notice that it is actually much more abundant than C7-8 because it
accumulates from isomeric transitions of \ceight\ and unlike $C_n$ does not
suffer from fast oxidation \citep{1999Sci...283.1290C}. Each of these
features is understandable in terms of the flows into and out of each species. 

Figure \ref{fig:cn} shows the shape of $C_n$ vs. $n$ for \cthree\ to
\ceight\ at two different temperatures near 2000K. As temperature declines,
the curve flattens because photoejection from $C_n$ by thermal photons
(or phonons) weakens. These abundance ratios are almost in a steady state,
but that steady state changes slightly as T falls. These patterns show
almost equal values for $Y(C_3)$ because that curve is relatively flat
with time near $t-t(6000K) = 10^7$ s (Fig. \ref{fig:yi}).
Table \ref{tab:flows} lists our computed reaction flows both in and out of
\csix\ at T = 2031K and T = 1802K. The reactions are specified there in the
compact notation target(in,out)residual. Fig. \ref{fig:cn} and
Table \ref{tab:flows} are made to be studied together.
Note that the abundances of \ceight\ are very small:
$Y(C_8)= 10^{-36}$ at T = 1802K. The isomeric transition from linear to
ringed \ceight, for which we estimate a lifetime $\tau = 10$ s,
provides the nucleations for graphite growth. Figure \ref{fig:ncoc2} showed
that $Y(C_{8}^r)$ 

We call attention to several conclusions to be drawn from Table \ref{tab:flows}.
Firstly, the strongest flows by a wide margin involving C6 are
$C_5 + C \to C_6$,
which is very fast ($k = 3 \times 10^{-10} cm^3 s^{-1}$)
because vibrational excitation of \csix\
obviates the need for creating a photon \citep{1999Sci...283.1290C},
and its inverse reaction which ejects a C atom. Furthermore, those flows
are equal to three significant figures, illustrating the near steady
state of the abundance pattern. Secondly, thermal dissociation of \csix\ is
very much faster than its oxidation, showing the small effect of oxidation
on the abundance pattern. The destroying flows from \csix\ to \cfive\
stand in the approximate ratio $C_6(\gamma,C)/C_6(O,CO) = 5 \times 10^5$
at $T = 2031$ K and $2 \times 10^4$ at $T = 1802$ K.
Photodissociation dominates. The smaller value for that ratio at T=1802K
occurs because oxidation maintains its effectiveness as T drops but
thermal disruption does not, being much more temperature sensitive.
For this reason the abundance pattern is flatter and C6 is approximately
3000 times more abundant at T=1802K than at T=2031K.
It is for that reason that destruction flows in Table \ref{tab:flows}
for C6 are
larger at the smaller temperature.
The steady state shifts to new ratios as T falls.
Thirdly, calculation of the rates will be illustrated for sake of clarity
by the flow $C_6 + O \to CO + C_5$.
From Fig. \ref{fig:cn}
one sees that at T=2031K the abundance $Y(C_6) = 10^{-33.3}$.
The value of $Y(O) = 10^{10} cm^{-3} / 1.1 \times 10^{10} cm^{-3} = 0.909$,
so the flow per atom per
second (e.g. Eq.3) is dY/dt= Y(C6)Y(O)n(2031K)k(C6O).
Expansion from T=6000K to T=2031K has diluted the total number
density n to 1.1x1010cm-3 (2031K/6000K)3= 4.27x108cm-3.
The fast oxidation reaction rate is  k(C6O)=3x10-10cm3s-1.
Gathering factors yields approximately 4x10-35 n-1s-1,
in good approximation to the Table \ref{tab:flows}
flow entry $3.61 \times 10^{-35} n^{-1}s^{-1}$. 

Our calculations have shown clearly that producing \ctwo\ molecules near
5000K via the CO + C reaction \citep{2009ApJ...703..642C} is not a viable
prospect for condensation of carbon dust via linear carbon chains.
The abundance of CO is far too small for it to seed high-T \ctwo\ production.
The abundance of C8 rings grows much later (Fig. 4) near 2000K, as
\citet{1999Sci...283.1290C,2001ApJ...562..480C} had found.
These rings grew from \ctwo\ created by simple carbon association,
$C + C \to C_2 + \gamma$.
Furthermore, the abundance of \ctwo\ is very small near 5000K
(Fig. \ref{fig:ncoc2}),
primarily because its thermal dissociation rate is much too fast for
it to have significant abundance at that high T.
What little \ctwo\ is made at 5000K is immediately dissociated by
thermal photos. Therefore its small steady-state abundance is inadequate
for building abundances of \cthree\ and beyond. 

\section{Contrast with \citet{2009ApJ...703..642C}}

The large numerical differences between the results of
\citet{1999Sci...283.1290C,2001ApJ...562..480C}
and those of \citet{2009ApJ...703..642C}
seem to be characterized by these differences:
\begin{enumerate}

\item Instead of growing large CO abundance near Y(CO)=0.1 at T=5500K
as in Fig. 11 of \citet{2009ApJ...703..642C}, we find that $Y(CO)$
builds to a maximum of only 10-3, which it achieves only slowly
(Fig. \ref{fig:ncoc2}), reaching that maximum only at T=3500K rather
than 5500K.

\item We find a maximum number density $n_{C_2} = 0.1$ cm$^{-3}$,
only about $10^{-7}$ of
$n_{CO}$. We find \ctwo\ to be rare at high temperature because its
dissociation rate by thermal photons is very much faster than for CO
owing to the smaller binding energy (6.3eV) of \ctwo. This $n_{C_2}$
maximum differs greatly from results in \citet{2009ApJ...703..642C},
where their Fig. 11 shows the maximum atom fraction of $n_{C_2}$ to be
near $10^{-5}$, equal to about $10^5$ cm$^{-3}$, almost 1\% of
$n_{CO}$ at that time. Our \ctwo/CO abundance ratio is,
in other words, only $10^{-5}$ of the same ratio calculated by
\citet{2009ApJ...703..642C}.

\end{enumerate}
We reasoned that these big differences could be understood if
\citet{2009ApJ...703..642C} had inadvertantly omitted the thermal
dissociation rates of small carbon molecules. Believing that to be the
cause for the discrepancy, we tested that hypothesis by performing our
own trial calculation involving only C, O, CO, \ctwo, and \cthree,
as they had done,
and omitted the $\tau_\gamma$ destruction terms.
Figure 6 displays the result.
The rapid rise of $Y(CO)$ to $10^{-2}$ at high temperature is very similar to
Fig. 11 of \citet{2009ApJ...703..642C}.
Similarity also describes $Y(C_2)$,
which grows quickly (Fig. 6) to 2x10-4 of Y(CO),
whereas our Fig. \ref{fig:ncoc2} shows the true value of $Y(CO)$ to be
only $10^{-8}$ at T=5500K and, considerably later,
$Y(C_2)/Y(CO) = 10^{-7}$ at their maxima. 

With photodissociation turned off, the destruction of \ctwo\ is primarily
by oxidation, $C_2+O \to CO + C$, whereas production of \ctwo is by
$CO + C \to C_2 + O$, the reaction we study in this work.
Those strive to balance, which if achieved establishes a steady state
ratio $Y(CO)/Y(C_2) = 3.5 \times 10^4$. Fig. 6 and their Fig. 11 do show
approximately that value, but $Y(C_2)$ declines faster than $Y(CO)$
because of declining production of \ctwo\ as T declines.

We could not expect detailed agreement, even if our hypothesis for the
cause of the discrepancy is correct. Our calculation used
O / C = 10 whereas they used O / C = 3 for the zone of their SNII model.
The temperature profiles also differ, we using T=3800/(t/100d) and they
T=18500/(t/100d)1.8.
We believe that their T is too hot owing to omission of CO cooling
(\citealp{1996ApJ...471..480L}, Fig. 5),
which we tried to accommodate roughly by
the choice T=3800K at t=100d, which is about 50% of the temperature
within published models which do omit CO cooling. And owing to the
factor $t^{-1.8}$, their T falls through a specified temperature drop
(say, 5000K to 3000K) more quickly than does our parameterization.
Nonetheless, any T profile declines though 5000K and reaches 3000K
somewhat later, so basically similar abundance results are expected. 

Such detailed differences are small in comparison with inclusion of
thermal photodissociation. The similarity of Fig. 6 to their Fig.11,
and the huge differences of these figures from those of our network
amounts in our minds to a resolution of the discrepancy. 

\acknowledgments

We are grateful.

\clearpage

\begin{figure}
\epsscale{.80}
%\plotone{f1.eps}
\caption{
Comparison of the density of CO molecules calculated by the numerical
network (curve) with the expectation nCOγeq = nCnO$k_{CO}$τγ for an abundance
nCO in thermal equilibrium (dots) with photons.
Above T=4000K the CO abundance is seen to be in an accurate thermal
equilibrium, but at lower temperature nCO is increasingly smaller than
thermal equilibrium would require. Instead of thermal photons,
the destruction of CO becomes increasingly dominated by
Compton electrons for $T < 3500$ K.
One see in Fig. \ref{fig:nco} that nCOmax occurs near 3500K,
which occurs near $10^7$ s (see Table \ref{tab:quantities})}. 
\label{fig:nco}
\end{figure}

\clearpage

\begin{figure}
\epsscale{.80}
%\plotone{f1.eps}
\caption{
Flows into and out of CO illustrate the near steady state between creation
and destruction and the transition region near 3500K between the two modes
of destruction of CO. The destruction flow nCOnC$k_{CO}$+C is only $3.27 \times
10^{-16}$ atom$^{-1}$s$^{-1}$, too small to be visible in Fig. \ref{fig:flows}. 
} \label{fig:flows}
\end{figure}

\clearpage

\begin{figure}
\epsscale{.80}
%\plotone{f1.eps}
\caption{
Number densities of CO and of \ctwo\ as functions of temperature.
The maximum density of \ctwo\ is 0.1 cm$^{-3}$.
Both abundances grow much more slowly than in the calculation by
\citet{2009ApJ...703..642C}.} \label{fig:ncoc2}
\end{figure}

\clearpage

\begin{figure}
\epsscale{.80}
%\plotone{f1.eps}
\caption{
The abundance (number per atom) of linear chains $C_n$ and of ringed isomer
\ceightr. We have scaled each $Y(C_n)$ by the factor stated in the box.}
\label{fig:yi}
\end{figure}

\clearpage

\begin{figure}
\epsscale{.80}
%\plotone{f1.eps}
\caption{
Shape of $Y(C_n)$ vs. n for \cthree to \ceight\ at two different temperatures
near 2000K. These shapes can be understood by the relative magnitudes of
flows into and out of $C_n$.
}
\label{fig:cn}
\end{figure}

\clearpage

\begin{deluxetable}{ccrrrrrrrrcrl}
\tabletypesize{\scriptsize}
\rotate
\tablecaption{Values of $\tau_\gamma$(CO) and related quantities at
selected $T$}
\label{tab:quantities}
\tablewidth{0pt}
\tablehead{
\colhead{POS} & \colhead{chip} & \colhead{ID} & \colhead{X} & \colhead{Y} &
\colhead{RA} & \colhead{DEC} & \colhead{IAU$\pm$ $\delta$ IAU} &
\colhead{IAP1$\pm$ $\delta$ IAP1} & \colhead{IAP2 $\pm$ $\delta$ IAP2} &
\colhead{star} & \colhead{E} & \colhead{Comment}
}
\startdata
0 & 2 & 1 & 1370.99 & 57.35    &   6.651120 &  17.131149 & 21.344$\pm$0.006  & 2
4.385$\pm$0.016 & 23.528$\pm$0.013 & 0.0 & 9 & -    \\
0 & 2 & 2 & 1476.62 & 8.03     &   6.651480 &  17.129572 & 21.641$\pm$0.005  & 2
3.141$\pm$0.007 & 22.007$\pm$0.004 & 0.0 & 9 & -    \\
0 & 2 & 3 & 1079.62 & 28.92    &   6.652430 &  17.135000 & 23.953$\pm$0.030  & 2
4.890$\pm$0.023 & 24.240$\pm$0.023 & 0.0 & - & -    \\
0 & 2 & 4 & 114.58  & 21.22    &   6.655560 &  17.148020 & 23.801$\pm$0.025  & 2
5.039$\pm$0.026 & 24.112$\pm$0.021 & 0.0 & - & -    \\
0 & 2 & 5 & 46.78   & 19.46    &   6.655800 &  17.148932 & 23.012$\pm$0.012  & 2
3.924$\pm$0.012 & 23.282$\pm$0.011 & 0.0 & - & -    \\
0 & 2 & 6 & 1441.84 & 16.16    &   6.651480 &  17.130072 & 24.393$\pm$0.045  & 2
6.099$\pm$0.062 & 25.119$\pm$0.049 & 0.0 & - & -    \\
0 & 2 & 7 & 205.43  & 3.96     &   6.655520 &  17.146742 & 24.424$\pm$0.032  & 2
5.028$\pm$0.025 & 24.597$\pm$0.027 & 0.0 & - & -    \\
0 & 2 & 8 & 1321.63 & 9.76     &   6.651950 &  17.131672 & 22.189$\pm$0.011  & 2
4.743$\pm$0.021 & 23.298$\pm$0.011 & 0.0 & 4 & edge \\
\enddata
\end{deluxetable}

%% If you use the table environment, please indicate horizontal rules using
%% \tableline, not \hline.
%% Do not put multiple tabular environments within a single table.
%% The optional \label should appear inside the \caption command.

\clearpage

\begin{table}
\begin{center}
\caption{Reaction Flows ($n^{-1}s^{-1}$) involving \csix\ at two temperatures.}
\label{tab:flows}
\begin{tabular}{crrrrrrrrrrr}
\tableline\tableline
Star & Height & $d_{x}$ & $d_{y}$ & $n$ & $\chi^2$ & $R_{maj}$ & $R_{min}$ &
\multicolumn{1}{c}{$P$\tablenotemark{a}} & $P R_{maj}$ & $P R_{min}$ &
\multicolumn{1}{c}{$\Theta$\tablenotemark{b}} \\
\tableline
1 &33472.5 &-0.1 &0.4  &53 &27.4 &2.065  &1.940 &3.900 &68.3 &116.2 &-27.639\\
2 &27802.4 &-0.3 &-0.2 &60 &3.7  &1.628  &1.510 &2.156 &6.8  &7.5 &-26.764\\
3 &29210.6 &0.9  &0.3  &60 &3.4  &1.622  &1.551 &2.159 &6.7  &7.3 &-40.272\\
4 &32733.8 &-1.2\tablenotemark{c} &-0.5 &41 &54.8 &2.282  &2.156 &4.313 &117.4 &78.2 &-35.847\\
5 & 9607.4 &-0.4 &-0.4 &60 &1.4  &1.669\tablenotemark{c}  &1.574 &2.343 &8.0  &8.9 &-33.417\\
6 &31638.6 &1.6  &0.1  &39 &315.2 & 3.433 &3.075 &7.488 &92.1 &25.3 &-12.052\\
\tableline
\end{tabular}
\end{center}
\end{table}

\bibliographystyle{apj}
\bibliography{clemson}

\end{document}

%%
%% End of file `sample.tex'.
